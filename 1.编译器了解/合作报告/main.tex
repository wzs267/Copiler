% -*- coding: utf-8 -*-
%-------------------------designed by zcf--------------
\documentclass[UTF8,a4paper,10pt]{ctexart}
\usepackage[left=3.17cm, right=3.17cm, top=2.74cm, bottom=2.74cm]{geometry}
\usepackage{amsmath}
\usepackage{graphicx,subfig}
\usepackage{float}
\usepackage{cite}
\usepackage{caption}
\usepackage{enumerate}
\usepackage{booktabs} %表格
\usepackage{multirow}
\newcommand{\tabincell}[2]{\begin{tabular}{@{}#1@{}}#2\end{tabular}}  %表格强制换行
%-------------------------字体设置--------------
% \usepackage{times} 
\usepackage{ctex}
\setCJKmainfont[ItalicFont=Noto Sans CJK SC Bold, BoldFont=Noto Serif CJK SC Black]{Noto Serif CJK SC}
\newcommand{\yihao}{\fontsize{26pt}{36pt}\selectfont}           % 一号, 1.4 倍行距
\newcommand{\erhao}{\fontsize{22pt}{28pt}\selectfont}          % 二号, 1.25倍行距
\newcommand{\xiaoer}{\fontsize{18pt}{18pt}\selectfont}          % 小二, 单倍行距
\newcommand{\sanhao}{\fontsize{16pt}{24pt}\selectfont}  %三号字
\newcommand{\xiaosan}{\fontsize{15pt}{22pt}\selectfont}        % 小三, 1.5倍行距
\newcommand{\sihao}{\fontsize{14pt}{21pt}\selectfont}            % 四号, 1.5 倍行距
\newcommand{\banxiaosi}{\fontsize{13pt}{19.5pt}\selectfont}    % 半小四, 1.5倍行距
\newcommand{\xiaosi}{\fontsize{12pt}{18pt}\selectfont}            % 小四, 1.5倍行距
\newcommand{\dawuhao}{\fontsize{11pt}{11pt}\selectfont}       % 大五号, 单倍行距
\newcommand{\wuhao}{\fontsize{10.5pt}{15.75pt}\selectfont}    % 五号, 单倍行距
%-------------------------章节名----------------
\usepackage{ctexcap} 
\CTEXsetup[name={,、},number={ \chinese{section}}]{section}
\CTEXsetup[name={(,)},number={\chinese{subsection}}]{subsection}
\CTEXsetup[name={,.},number={\arabic{subsubsection}}]{subsubsection}
%-------------------------页眉页脚--------------
\usepackage{fancyhdr}
\pagestyle{fancy}
\lhead{\kaishu \leftmark}
% \chead{}
\rhead{\kaishu 语言处理系统的完整工作过程实验报告}%加粗\bfseries 
\lfoot{}
\cfoot{\thepage}
\rfoot{}
\renewcommand{\headrulewidth}{0.1pt}  
\renewcommand{\footrulewidth}{0pt}%去掉横线
\newcommand{\HRule}{\rule{\linewidth}{0.5mm}}%标题横线
\newcommand{\HRulegrossa}{\rule{\linewidth}{1.2mm}}
%-----------------------伪代码------------------
\usepackage{algorithm}  
\usepackage{algorithmicx}  
\usepackage{algpseudocode}  
\floatname{algorithm}{Algorithm}  
\renewcommand{\algorithmicrequire}{\textbf{Input:}}  
\renewcommand{\algorithmicensure}{\textbf{Output:}} 
\usepackage{lipsum}  
\makeatletter
\newenvironment{breakablealgorithm}
  {% \begin{breakablealgorithm}
  \begin{center}
     \refstepcounter{algorithm}% New algorithm
     \hrule height.8pt depth0pt \kern2pt% \@fs@pre for \@fs@ruled
     \renewcommand{\caption}[2][\relax]{% Make a new \caption
      {\raggedright\textbf{\ALG@name~\thealgorithm} ##2\par}%
      \ifx\relax##1\relax % #1 is \relax
         \addcontentsline{loa}{algorithm}{\protect\numberline{\thealgorithm}##2}%
      \else % #1 is not \relax
         \addcontentsline{loa}{algorithm}{\protect\numberline{\thealgorithm}##1}%
      \fi
      \kern2pt\hrule\kern2pt
     }
  }{% \end{breakablealgorithm}
     \kern2pt\hrule\relax% \@fs@post for \@fs@ruled
  \end{center}
  }
\makeatother
%------------------------代码-------------------
\usepackage{xcolor} 
\usepackage{listings} 
\lstset{ 
breaklines,%自动换行
basicstyle=\small,
escapeinside=``,
keywordstyle=\color{ blue!70} \bfseries,
commentstyle=\color{red!50!green!50!blue!50},% 
stringstyle=\ttfamily,% 
extendedchars=false,% 
linewidth=\textwidth,% 
numbers=left,% 
numberstyle=\tiny \color{blue!50},% 
frame=trbl% 
rulesepcolor= \color{ red!20!green!20!blue!20} 
}
%------------超链接----------
\usepackage[colorlinks,linkcolor=black,anchorcolor=blue]{hyperref}
%------------------------TODO-------------------
\usepackage{enumitem,amssymb}
\newlist{todolist}{itemize}{2}
\setlist[todolist]{label=$\square$}
% for check symbol 
\usepackage{pifont}
\newcommand{\cmark}{\ding{51}}%
\newcommand{\xmark}{\ding{55}}%
\newcommand{\done}{\rlap{$\square$}{\raisebox{2pt}{\large\hspace{1pt}\cmark}}\hspace{-2.5pt}}
\newcommand{\wontfix}{\rlap{$\square$}{\large\hspace{1pt}\xmark}}
%------------------------水印-------------------
\usepackage{tikz}
\usepackage{xcolor}
\usepackage{eso-pic}

\newcommand{\watermark}[3]{\AddToShipoutPictureBG{
\parbox[b][\paperheight]{\paperwidth}{
\vfill%
\centering%
\tikz[remember picture, overlay]%
  \node [rotate = #1, scale = #2] at (current page.center)%
    {\textcolor{gray!80!cyan!30!magenta!30}{#3}};
\vfill}}}



%———————————————————————————————————————————正文———————————————————————————————————————————————
%----------------------------------------------
\begin{document}
\begin{titlepage}
    \begin{center}
    \includegraphics[width=0.8\textwidth]{NKU.png}\\[1cm]    
    \textsc{\Huge \kaishu{\textbf{南\ \ \ \ \ \ 开\ \ \ \ \ \ 大\ \ \ \ \ \ 学}} }\\[0.9cm]
    \textsc{\huge \kaishu{\textbf{计\ \ 算\ \ 机\ \ 学\ \ 院}}}\\[0.5cm]
    \textsc{\Large \textbf{语言处理系统的完整工作过程实验报告}}\\[0.8cm]
    \HRule \\[0.9cm]
    { \LARGE \bfseries 语言处理系统探究}\\[0.4cm]
    \HRule \\[2.0cm]
    \centering
    \textsc{\LARGE 杨乔钦{\ \ \ \ }  王泽舜{\ \ \ \ }}\\[0.5cm]
    \textsc{\LARGE \kaishu{年级\ :\ 2023级}}\\[0.5cm]
    \textsc{\LARGE \kaishu{专业\ :\ 计算机科学与技术}}\\[0.5cm]
    \textsc{\LARGE \kaishu{指导教师\ :\ 王刚}}\\[0.5cm]
    \vfill
    {\Large \today}
    \end{center}
\end{titlepage}
%-------------摘------要--------------
\newpage
\thispagestyle{empty}
\renewcommand{\abstractname}{\kaishu \sihao \textbf{摘要}}
    \begin{abstract}

        \noindent  %顶格
        \textbf{\\\ 关键字:Parallel}\textbf{} \\\ \\\
    \end{abstract}
%----------------------------------------------------------------
\tableofcontents
%----------------------------------------------------------------
\newpage
\watermark{60}{10}{NKU}
\setcounter{page}{1}
\section{原始SysY程序}
我们编写了一个简单的SysY程序,功能是读取一个整数数组,计算其元素和,并根据和的大小进行不同的输出操作
如果大于20,会输出和的两倍,如果小于等于20,会输出和除以3的余数。
程序中包含了变量声明、数组操作、算术运算、条件判断和函数调用等基本语法结构。以下是该SysY程序的完整代码:
\begin{lstlisting}[title=逐列访问平凡算法,frame=trbl,language={C}]
// 全局变量和常量声明
const int N = 5;
int global_array[5] = {2,2,3,4,5};

// SysY运行时库函数声明 - 必须添加这些声明才能调用库函数
void putint(int);       // 输出一个整数
void putch(int);        // 输出一个字符

// 函数声明
int calculate(int a, int b);

int main() {

    int sum = 0;
    int i = 0;
    
    // 使用getarray从输入获取数据,替换原来的固定数组赋值
    int count = 5;

    
    // 算术运算:计算数组元素的和
    while (i < count) {  // 使用实际读取的元素个数
        sum = sum + global_array[i];
        i = i + 1;
    }
    
    // 条件判断和输出
    if (sum > 20) {
        int result = calculate(sum, 2);
    putint(result);  // 使用putint输出结果而不是直接return
    putch(10);      // 输出换行符
    } else {
        int remainder = sum % 3;
    putint(remainder); // 使用putint输出结果
    putch(10);        // 输出换行符
    }
    
    return 0;  // main函数返回0表示正常结束
}

// 函数定义:复杂计算
int calculate(int a, int b) {
    int result = 0;
    while (b > 0) {
        result = result + a;
        b = b - 1;
    }
    return result;
}\end{lstlisting}


\section{arm汇编程序}
我编写的arm汇编程序如下所示,该程序实现了与上述SysY程序相同的功能。它读取一个整数数组,计算其元素和,并根据和的大小进行不同的输出操作。如果和大于20,则输出和的两倍;如果和小于等于20,则输出和除以3的余数。程序中包含了变量声明、数组操作、算术运算、条件判断和函数调用等基本语法结构。
以下是该arm汇编程序的完整代码:
\begin{lstlisting}[title=逐列访问平凡算法,frame=trbl]
	.arch armv8-a ; 指定目标架构为 AArch64
	.file	"test.c" ; 源文件名信息
	.text ; 文本段开始
	.section	.text.startup,"ax",@progbits ; 启动代码段
	.align	2 ; 对齐
	.p2align 4,,11 ; 更高对齐
	.global	main ; 导出 main 符号
	.type	main, %function ; 声明 main 为函数
main: ; main 函数入口
.LFB0: ; 函数框架开始标签
	.cfi_startproc ; 调试信息开始
	adrp	x0, .LANCHOR0 ; 将 LAUNCHER 基址高位加载到 x0
	add	x1, x0, :lo12:.LANCHOR0 ; 将基址低位加到 x0 得到完整地址放 x1
	stp	x29, x30, [sp, -16]! ; 保存帧指针和返回地址并调整栈(push)
	.cfi_def_cfa_offset 16 ; 调试: 定义 CFA 偏移
	.cfi_offset 29, -16 ; 调试: x29 保存位置
	.cfi_offset 30, -8 ; 调试: x30 保存位置
	mov	x29, sp ; 设置帧指针 x29 = sp
	ldp	w0, w4, [x1] ; 从内存载入两个 32-bit 值到 w0,w4
	ldp	w3, w2, [x1, 8] ; 继续加载另外两个 32-bit 值到 w3,w2
	add	w0, w0, w4 ; 累加 w0 和 w4
	ldr	w1, [x1, 16] ; 载入第五个 32-bit 值到 w1
	add	w0, w0, w3 ; 累加 w3
	add	w0, w0, w2 ; 累加 w2
	add	w0, w0, w1 ; 累加 w1,w0 保存元素和
	cmp	w0, 20 ; 比较和与 20
	ble	.L2 ; 若小于等于 20 跳到 .L2
	lsl	w0, w0, 1 ; 否则将和乘 2(左移一位)
	bl	putint ; 调用 putint 输出整数
	mov	w0, 10 ; 将换行符 ASCII 10 放入 w0
	bl	putch ; 调用 putch 输出换行
.L3: ; 公共返回标签
	mov	w0, 0 ; 返回值置 0
	ldp	x29, x30, [sp], 16 ; 恢复 x29,x30 并释放栈空间
	.cfi_remember_state ; 调试信息:记住状态
	.cfi_restore 30 ; 恢复调试寄存器 30 信息
	.cfi_restore 29 ; 恢复调试寄存器 29 信息
	.cfi_def_cfa_offset 0 ; 恢复 CFA 偏移
	ret ; 返回调用者
.L2: ; 小于等于20的处理分支
	.cfi_restore_state ; 恢复调试状态
	mov	w1, 3 ; 将除数 3 放入 w1
	sdiv	w1, w0, w1 ; w1 = w0 / 3(商)
	add	w1, w1, w1, lsl 1 ; w1 = w1 + (w1 << 1) = 3 * 商
	sub	w0, w0, w1 ; 计算余数 w0 = w0 - 3*商
	bl	putint ; 输出余数
	mov	w0, 10 ; 设置换行字符
	bl	putch ; 输出换行
	b	.L3 ; 跳转到公共返回
	.cfi_endproc ; 结束调试信息
.LFE0: ; 函数结束标签
	.size	main, .-main ; 记录 main 大小
	.text ; 文本段(下一个函数)
	.align	2 ; 对齐
	.p2align 4,,11 ; 对齐
	.global	calculate ; 导出 calculate 符号
	.type	calculate, %function ; 声明 calculate
calculate: ; calculate 函数入口
.LFB1: ; 函数框架开始
	.cfi_startproc ; 调试开始
	mul	w0, w1, w0 ; w0 = w0 * w1(乘法实现)
	cmp	w1, 0 ; 比较参数 w1 与 0
	csel	w0, w0, wzr, gt ; 若 w1>0 返回乘积,否则返回 0
	ret ; 返回
	.cfi_endproc ; 结束调试信息
.LFE1: ; 结束标签
	.size	calculate, .-calculate ; 记录大小
	.global	global_array ; 导出全局数组
	.global	N ; 导出常量 N
	.section	.rodata ; 只读数据段
	.align	2 ; 对齐
	.type	N, %object ; 声明 N 为对象
	.size	N, 4 ; N 的大小
N: ; 常量 N 标签
	.word	5 ; N = 5
	.data ; 数据段开始
	.align	4 ; 数据对齐
	.set	.LANCHOR0,. + 0 ; 设置锚点符号
	.type	global_array, %object ; 声明全局数组
	.size	global_array, 20 ; 数组大小 20 字节
global_array: ; 全局数组标签
	.word	2 ; 元素 0 = 2
	.word	2 ; 元素 1 = 2
	.word	3 ; 元素 2 = 3
	.word	4 ; 元素 3 = 4
	.word	5 ; 元素 4 = 5
	.ident	"GCC: (Ubuntu 13.3.0-6ubuntu2~24.04) 13.3.0" ; 编译器标识
	.section	.note.GNU-stack,"",@progbits ; 指示不可执行栈

\end{lstlisting}
\subsection{说明}
\begin{itemize}
	\item 调用约定:AArch64 使用寄存器 x0..x7 传递前八个整型/指针参数,返回值放在 x0(32-bit 时用 w0)。被调用者保存寄存器(如 x19..x28)需在函数入口保存并在返回前恢复。
	\item 栈帧与栈保护:函数入口处通过 \texttt{sub sp, sp, \#64} 分配栈空间,编译器加入了栈保护(\texttt{\_stack\_chk\_guard} / \texttt{\_stack\_chk\_fail})以检测栈溢出攻击。
	\item 数组寻址:编译器使用位移指令(例如 \texttt{lsl 2})将索引乘以 4 来计算整数元素的字节偏移,从而高效访问数组元素。
	\item 循环与分支优化:简单的循环被编译为比较/分支形式;小的数学变换(例如将循环展开或用乘法替代循环)可能由编译器进行优化,如本例中 \texttt{calculate} 使用乘法和条件选择实现语义。
	\item 运行时库依赖:汇编中调用的 \texttt{putint}、\texttt{putch} 等函数由运行时库(例如 \texttt{libsysy\_aarch.a})提供。链接时必须指定对应目标架构的静态库,否则会出现未定义引用错误。
	\item 链接建议:使用交叉链接器将汇编/目标文件与 AArch64 静态库链接,例如:
	\begin{verbatim}
aarch64-linux-gnu-gcc test_aarch64.s libsysy_aarch.a -o test_arm.out
	\end{verbatim}
	若库放在子目录则传递相对路径:\texttt{lib/..../libsysy\_aarch.a}。
	\item 架构匹配:务必确保静态库是为目标架构(AArch64)构建的;x86 构建的库无法用于 ARM 链接。
	\item 调试与反汇编:若需查看生成的可执行的真实指令或符号信息,可用 \texttt{objdump -d} / \texttt{readelf} / \texttt{nm} 等工具检查符号表与节信息。
\end{itemize}

\subsection{结果展示}
我们将上述SysY程序编译为ARM汇编,并链接运行时库,成功生成了可执行文件。以下是SysY程序和arm汇编的编译成功控制台信息:
\begin{lstlisting}[title=运行成功信息,frame=trbl]
yqq@LAPTOP-B4JFEDK6:~/repository/Copiler/1.编译器了解$ aarch64-linux-gnu-gcc test_aarch64.s lib.tar.gz/libsysy_aarch.a -o test_arm.out
yqq@LAPTOP-B4JFEDK6:~/repository/Copiler/1.编译器了解$ qemu-aarch64 -L /usr/aarch64-linux-gnu ./test_arm.out
1
TOTAL: 0H-0M-0S-0us
\end{lstlisting}
\begin{table}[!htbp]
  \centering
\begin{tabular}{ccc}
		oprule
	array内容 & SysY编译输出 & arm汇编编译输出 \\
	\midrule
	1,2,3,4,5 & 0 & 0 \\
	2,2,3,4,5 & 1 & 1 \\
	3,2,3,4,5 & 2 & 2 \\
	4,2,3,4,5 & 0 & 0 \\
	\bottomrule
\end{tabular}
  \caption{运行结果比较}
\end{table}
由此可见,SysY程序和arm汇编程序在相同输入下均能正确输出预期结果,功能实现一致。



%——————————————————————————————————————
% \subsection{第一节}
% 如图\ref{fig:1}所示
% \begin{figure}[H]
%     \centering
%     \includegraphics[scale=0.3]{NKU.png}
%     \caption{Caption}
%     \label{fig:1}
% \end{figure}

% 表
% \begin{table}[!htbp]
%   \centering
%   \begin{tabular}{ccccccccccc}
%   \toprule  
%   N/n$\backslash$Algo& naive-conv& naive-pool& omp-conv& omp-pool\\
%   \midrule
%   64/2& 0.0167& 0.01255& 0.04142& 0.03799\\
%   64/4& 0.03599&0.0394& 0.0458& 0.0421\\
%   \bottomrule
%   \end{tabular}
%   \caption{性能测试结果(4线程)(单位:ms)}
% \end{table}

% 带单元格表格
% \begin{table}[!htbp]
%   \centering
%   \begin{tabular}{|c|c|c|c|c|c|c|}
%   \hline
%   \multicolumn{2}{|c|}{ \multirow{2}*{$Cost$} }& \multicolumn{5}{c|}{To}\\
%   \cline{3-7}
%   \multicolumn{2}{|c|}{}&$A$&$B$&$C$&$D$&$E$\\
%   \hline
%   \multirow{3}*{From}&$B$&7&0&1&3&8\\
%   \cline{2-7}
%   &$C$&8&1&0&2&7\\
%   \cline{2-7}
%   &$D$&8&3&2&0&5\\
%   \hline
%   \end{tabular}
%   \caption{结点C距离向量表(无毒性逆转)}
% \end{table}

% %——————————————————————————————————————
% \subsection{第二节}
% 伪代码

% \begin{breakablealgorithm} 
%   \caption{初始化obj文件信息——对应MeshSimplify类中readfile函数,Face类calMatrix函数} 
%   \begin{algorithmic}[1] %每行显示行号  
%       \Require obj文件,顶点、边、面列表
%       \Ensure 是否读取成功
%       \Function {calMatrix}{$Face$}  
%               \State $normal \gets e1×e2$  
%               \State $normal \gets normal/normal.length$
%               \State $temp[] \gets {normal.x, normal.y, normal.z, normal· Face.v1}$
%               \State $Matrix[i][j]=temp[i] * temp[j]$ 
%               \State \Return{$Matrix$}  
%       \EndFunction
%       \State 根据obj的v和f区分点面信息,读取并加入列表
%       \State $scale \gets $记录点坐标中距离原点最远的分量,以便后续OpenGL进行显示
%       \State $ori \gets $记录中心点,便于OpenGL显示在中心位置,避免有的obj偏移原点较多
%       \State 根据三角面片信息,计算一个面的三条边
%       \State 计算每个面的矩阵$\gets calMatrix$
%       \State 将每个面的矩阵加到各点,由点维护\\
%       \Return True
%   \end{algorithmic}  
% \end{breakablealgorithm}

% 代码
% \begin{lstlisting}[title=逐列访问平凡算法,frame=trbl,language={C++}]
%   void ord()   
%   {
%       double head,tail,freq,head1,tail1,timess=0; // timers
%       init(N);
%       QueryPerformanceFrequency((LARGE_INTEGER *)&freq );
%       QueryPerformanceCounter((LARGE_INTEGER *)&head);
%       for (int i=0; i<NN; i++)
%           for (int j=0; j<NN; j++)
%               col_sum[i] += (b[j][i]*a[j]);
%       QueryPerformanceCounter ((LARGE_INTEGER *)& tail) ;
%       cout << "\nordCol:" <<(tail-head)*1000.0 / freq<< "ms" << endl;
%   }
% \end{lstlisting}


% %——————————————————————————————————————
% \subsection{第三节}

% 参考文献\cite{adams1995hitchhiker}\cite{shin2016deep}
    
% 多行公式
% \begin{align}
%   a+b = a + b \\
%   \frac{a+b}{a-b}
% \end{align}

% 行内公式:$\sum^N_{i=1}$

% \textbf{超链接}  \href{http://youtube.com/}{YouTube}

% 带标号枚举
% \begin{enumerate}
%   \item 1
%   \item 2
% \end{enumerate}

% 不带标号枚举
% \begin{itemize}
%   \item 1
%   \item 2
% \end{itemize}

% \xiaosi{切换字体大小}

% %----------------------------------------------------------------
% \section{总结}

% %----------------------------------------------------------------
\newpage
\bibliographystyle{plain}
\bibliography{references} 
\end{document}
