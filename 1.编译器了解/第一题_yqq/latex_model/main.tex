% -*- coding: utf-8 -*-
%-------------------------designed by zcf--------------
\documentclass[UTF8,a4paper,10pt]{ctexart}
\usepackage[left=3.17cm, right=3.17cm, top=2.74cm, bottom=2.74cm]{geometry}
\usepackage{amsmath}
\usepackage{graphicx,subfig}
\usepackage{float}
\usepackage{cite}
\usepackage{caption}
\usepackage{enumerate}
\usepackage{booktabs} %表格
\usepackage{multirow}
\usepackage{array}
\usepackage{tabularx}
\newcommand{\tabincell}[2]{\begin{tabular}{@{}#1@{}}#2\end{tabular}}  %表格强制换行
%-------------------------字体设置--------------
% \usepackage{times} 
\usepackage{ctex}
\setCJKmainfont[ItalicFont=Noto Sans CJK SC Bold, BoldFont=Noto Serif CJK SC Black]{Noto Serif CJK SC}
\newcommand{\yihao}{\fontsize{26pt}{36pt}\selectfont}           % 一号, 1.4 倍行距
\newcommand{\erhao}{\fontsize{22pt}{28pt}\selectfont}          % 二号, 1.25倍行距
\newcommand{\xiaoer}{\fontsize{18pt}{18pt}\selectfont}          % 小二, 单倍行距
\newcommand{\sanhao}{\fontsize{16pt}{24pt}\selectfont}  %三号字
\newcommand{\xiaosan}{\fontsize{15pt}{22pt}\selectfont}        % 小三, 1.5倍行距
\newcommand{\sihao}{\fontsize{14pt}{21pt}\selectfont}            % 四号, 1.5 倍行距
\newcommand{\banxiaosi}{\fontsize{13pt}{19.5pt}\selectfont}    % 半小四, 1.5倍行距
\newcommand{\xiaosi}{\fontsize{12pt}{18pt}\selectfont}            % 小四, 1.5倍行距
\newcommand{\dawuhao}{\fontsize{11pt}{11pt}\selectfont}       % 大五号, 单倍行距
\newcommand{\wuhao}{\fontsize{10.5pt}{15.75pt}\selectfont}    % 五号, 单倍行距
%-------------------------章节名----------------
\usepackage{ctexcap} 
\CTEXsetup[name={,、},number={ \chinese{section}}]{section}
\CTEXsetup[name={(,)},number={\chinese{subsection}}]{subsection}
\CTEXsetup[name={,.},number={\arabic{subsubsection}}]{subsubsection}
%-------------------------页眉页脚--------------
\usepackage{fancyhdr}
\pagestyle{fancy}
\lhead{\kaishu \leftmark}
% \chead{}
\rhead{\kaishu 语言处理系统的完整工作过程实验报告}%加粗\bfseries 
\lfoot{}
\cfoot{\thepage}
\rfoot{}
\renewcommand{\headrulewidth}{0.1pt}  
\renewcommand{\footrulewidth}{0pt}%去掉横线
\newcommand{\HRule}{\rule{\linewidth}{0.5mm}}%标题横线
\newcommand{\HRulegrossa}{\rule{\linewidth}{1.2mm}}
%-----------------------伪代码------------------
\usepackage{algorithm}  
\usepackage{algorithmicx}  
\usepackage{algpseudocode}  
\floatname{algorithm}{Algorithm}  
\renewcommand{\algorithmicrequire}{\textbf{Input:}}  
\renewcommand{\algorithmicensure}{\textbf{Output:}} 
\usepackage{lipsum}  
\makeatletter
\newenvironment{breakablealgorithm}
  {% \begin{breakablealgorithm}
  \begin{center}
     \refstepcounter{algorithm}% New algorithm
     \hrule height.8pt depth0pt \kern2pt% \@fs@pre for \@fs@ruled
     \renewcommand{\caption}[2][\relax]{% Make a new \caption
      {\raggedright\textbf{\ALG@name~\thealgorithm} ##2\par}%
      \ifx\relax##1\relax % #1 is \relax
         \addcontentsline{loa}{algorithm}{\protect\numberline{\thealgorithm}##2}%
      \else % #1 is not \relax
         \addcontentsline{loa}{algorithm}{\protect\numberline{\thealgorithm}##1}%
      \fi
      \kern2pt\hrule\kern2pt
     }
  }{% \end{breakablealgorithm}
     \kern2pt\hrule\relax% \@fs@post for \@fs@ruled
  \end{center}
  }
\makeatother
%------------------------代码-------------------
\usepackage{xcolor} 
\usepackage{listings} 
\lstset{ 
breaklines,%自动换行
basicstyle=\small,
escapeinside=``,
keywordstyle=\color{ blue!70} \bfseries,
commentstyle=\color{red!50!green!50!blue!50},% 
stringstyle=\ttfamily,% 
extendedchars=false,% 
linewidth=\textwidth,% 
numbers=left,% 
numberstyle=\tiny \color{blue!50},% 
frame=trbl% 
rulesepcolor= \color{ red!20!green!20!blue!20} 
}
%------------超链接----------
\usepackage[colorlinks,linkcolor=black,anchorcolor=blue]{hyperref}
%------------------------TODO-------------------
\usepackage{enumitem,amssymb}
\newlist{todolist}{itemize}{2}
\setlist[todolist]{label=$\square$}
% for check symbol 
\usepackage{pifont}
\newcommand{\cmark}{\ding{51}}%
\newcommand{\xmark}{\ding{55}}%
\newcommand{\done}{\rlap{$\square$}{\raisebox{2pt}{\large\hspace{1pt}\cmark}}\hspace{-2.5pt}}
\newcommand{\wontfix}{\rlap{$\square$}{\large\hspace{1pt}\xmark}}
%------------------------水印-------------------
\usepackage{tikz}
\usepackage{xcolor}
\usepackage{eso-pic}

\newcommand{\watermark}[3]{\AddToShipoutPictureBG{
\parbox[b][\paperheight]{\paperwidth}{
\vfill%
\centering%
\tikz[remember picture, overlay]%
  \node [rotate = #1, scale = #2] at (current page.center)%
    {\textcolor{gray!80!cyan!30!magenta!30}{#3}};
\vfill}}}



%———————————————————————————————————————————正文———————————————————————————————————————————————
%----------------------------------------------
\begin{document}
\begin{titlepage}
    \begin{center}
    \includegraphics[width=0.8\textwidth]{NKU.png}\\[1cm]    
    \textsc{\Huge \kaishu{\textbf{南\ \ \ \ \ \ 开\ \ \ \ \ \ 大\ \ \ \ \ \ 学}} }\\[0.9cm]
    \textsc{\huge \kaishu{\textbf{计\ \ 算\ \ 机\ \ 学\ \ 院}}}\\[0.5cm]
    \textsc{\Large \textbf{语言处理系统的完整工作过程实验报告}}\\[0.8cm]
    \HRule \\[0.9cm]
    { \LARGE \bfseries 语言处理系统探究}\\[0.4cm]
    \HRule \\[2.0cm]
    \centering
    \textsc{\LARGE 杨乔钦\kaishu{\ \ \ \ }}\\[0.5cm]
    \textsc{\LARGE \kaishu{年级\ :\ 2023级}}\\[0.5cm]
    \textsc{\LARGE \kaishu{专业\ :\ 计算机科学与技术}}\\[0.5cm]
    \textsc{\LARGE \kaishu{指导教师\ :\ 王刚}}\\[0.5cm]
    \vfill
    {\Large \today}
    \end{center}
\end{titlepage}
%-------------摘------要--------------
\newpage
\thispagestyle{empty}
\renewcommand{\abstractname}{\kaishu \sihao \textbf{摘要}}
    \begin{abstract}
123
        \noindent  %顶格
        \textbf{\\\ 关键字:Parallel}\textbf{} \\\ \\\
    \end{abstract}
%----------------------------------------------------------------
\tableofcontents
%----------------------------------------------------------------
\newpage
\watermark{60}{10}{NKU}
\setcounter{page}{1}
\section{预处理器}

\# <数字> <文件名> <标志>这是编译器预处理器产生的注释(注释行),用于追踪源代码中每一行的来源,便于调试、错误追踪和调试信息生成。
例如:$$\# 1 "test.cpp"$$表示接下来要处理的代码来源于 "test.cpp",且在原始源文件 test.cpp 中,从 第 1 行 开始。
$$\# 37 "/usr/include/c++/13/iostream" 3$$表示接下来的代码来自于 <iostream> 文件;原始 iostream 文件的第 37 行;3 表示这是一个系统头文件。
\par
之后文件都是处理头文件中的内容,将各个头文件中包含的头文件进行处理。
直到test.i文件的末尾,即文件的第36583行,我们才开始处理源文件test.cpp中的第二行内容,
之后便全是源文件test.cpp中的内容了。
\par
预处理器的主要任务包括:
\begin{itemize}
  \item 宏定义和展开:处理宏定义(如 \#define)并在代码中展开宏。
  \item 文件包含:处理文件包含指令(如 \#include),将包含的文件内容插入到当前文件中。
  \item 条件编译:处理条件编译指令(如 \#ifdef、\#ifndef、\#if、\#else、\#elif、\#endif),根据条件决定哪些代码被编译。
  \item 其他预处理指令:处理其他预处理指令,如 \#undef、\#pragma 等。
\end{itemize}
\par
预处理器的输出是一个纯文本文件,包含了所有预处理指令处理后的代码,这个文件随后会被编译器用来生成目标代码。
\par

\section{编译器}
\subsection{词法分析}
词法分析器的主要任务是将源代码转换为一系列的记号(tokens)。这些记号是源代码中有意义的基本单位,如关键字、标识
符、操作符、字面值等。词法分析器通过扫描源代码,识别出这些记号,并为每个记号分配一个类型和属性。
\par
词法分析器的工作流程通常包括以下几个步骤:
\begin{itemize}
  \item 输入读取:词法分析器从源代码文件中逐字符读取输入。
  \item 模式匹配:使用正则表达式或有限状态机(FSM)来识别不同类型的记号。
  \item 记号生成:当识别出一个记号时,词法分析器会创建一个记号对象,包含记号的类型、值和位置等信息。
  \item 忽略空白和注释:词法分析器通常会忽略源代码中的空白字符和注释,因为它们对程序的语义没有影响。
  \item 错误处理:如果遇到无法识别的字符或模式,词法分析器会报告词法错误。       
  \item 输出记号流:词法分析器将生成的记号按顺序输出,供后续的语法分析器使用。
\end{itemize}
\par
这些格式统一为<token\_type> '<literal>'  [属性...]  Loc=<文件名:行号:列号>
从前到后依次是词法单元类型、实际文本内容、可选属性、源文件中位置。
\par
在输入命令$clang -E -Xclang -dump-tokens main.c$后,词法分析器会输出源代码中的所有记号及其类型。
例如,输出中的$identifier$表示标识符,$keyword$表示关键字,$semi$表示分号等。并且所有内容都是test.i
中的内容,但是\#后面的语句不会被识别。
\subsection{语法分析}   
语法分析器的主要任务是将词法分析器生成的记号流转换为抽象语法树(AST)。抽象语法树是一种树状结构,表示源代码的语法结构和层次关系。 
\par
语法分析器的工作流程通常包括以下几个步骤:
\begin{itemize}
  \item 输入读取:语法分析器从词法分析器生成的记号流中读取输入。
  \item 语法规则:语法分析器使用一系列语法规则来解析记号流,生成抽象语法树。
  \item 错误处理:如果遇到无法识别的语法结构,语法分析器会报告语法错误。
  \item 输出抽象语法树:语法分析器将生成的抽象语法树输出,供后续的语义分析器使用。
  \item 抽象语法树的节点类型:抽象语法树的节点类型通常包括表达式节点、语句节点、声明节点等,每种节点类型表示不同的语法结构。
\end{itemize}
\par
Clang AST 每行节点的通用格式为:<树形符号><节点类型> <地址> <源码位置> <其他属性...>\par
其中不同类型节点的具体格式如下:
\subsubsection*{Decl 类节点(声明)}
\begin{tabular}{@{}ll@{}}
\toprule
字段名       & 示例/说明 \\ \midrule
节点类型     & \texttt{FunctionDecl}, \texttt{VarDecl}, \texttt{ParmVarDecl}, \texttt{UsingDirectiveDecl} \\
名字 + 类型 & \texttt{'int'}, \texttt{'int ()'}, \texttt{'ostream'} \\
使用状态     & \texttt{used}, \texttt{referenced} \\
初始化信息   & \texttt{cinit}, \texttt{callinit} \\
存储类       & \texttt{extern}, \texttt{static}(视情况而定) \\
\bottomrule
\end{tabular}

\vspace{1em}

\subsubsection*{Stmt 类节点(语句)}
\begin{tabular}{@{}ll@{}}
\toprule
字段名     & 示例/说明 \\ \midrule
节点类型   & \texttt{CompoundStmt}, \texttt{IfStmt}, \texttt{ReturnStmt}, \texttt{WhileStmt} \\
地址       & 有时省略 \\
位置信息   & 主要包含代码在源码中的位置 \\
名称字段   & 无(语句通常不带名称) \\
结构作用   & 用于建立语法树嵌套结构 \\
\bottomrule
\end{tabular}

\vspace{1em}

\subsubsection*{Expr 类节点(表达式)}
\begin{tabular}{@{}ll@{}}
\toprule
字段名       & 示例/说明 \\ \midrule
节点类型     & \texttt{BinaryOperator}, \texttt{DeclRefExpr}, \texttt{CallExpr}, \texttt{CXXOperatorCallExpr} \\
值类别       & \texttt{lvalue}, \texttt{xvalue}, \texttt{prvalue} \\
类型         & 表达式求值结果,如 \texttt{'int'} \\
操作符       & 如 \texttt{'+'}, \texttt{'*'}, \texttt{'='} \\
字面值       & 如 \texttt{IntegerLiteral 'int' 0} \\
隐式转换信息 & 如 \texttt{<FunctionToPointerDecay>}, \texttt{<LValueToRValue>} \\
\bottomrule
\end{tabular}

举例说明:
\begin{tabularx}{\textwidth}{>{\ttfamily}X}
|-UsingDirectiveDecl 0x617d83d7d7d8 <test.cpp:2:1, col:17> col:17 Namespace 0x617d82f2e528 'std'\\
\end{tabularx}
为例,UsingDirectiveDecl表示一个 using directive,即 using 指令;
0x617d83d7d7d8这是编译器给这个 AST 节点分配的内存地址(或唯一 id);
<test.cpp:2:1, col:17> col:17表示代码原位置;
Namespace 0x617d82f2e528 'std'表示这个 using 指令引入的命名空间是 std
,其节点位置在0x617d82f2e528,这里只是引用。
类似的,其他节点类型也有各自的格式。
\par

\subsection{语义分析}
语义分析器的主要任务是检查抽象语法树(AST)是否符合语言的语义规则,并进行类型检查、作用域管理和符号表维护等工作。
\par
语义分析器的工作流程通常包括以下几个步骤:
\begin{itemize}
  \item 输入读取:语义分析器从语法分析器生成的抽象语法树中读取输入。
  \item 语义规则:语义分析器使用一系列语义规则来检查抽象语法树,并进行类型检查、作用域管理和符号表维护。
  \item 错误处理:如果遇到语义错误,语义分析器会报告错误。
  \item 输出符号表:语义分析器将生成的符号表输出,供后续的中间代码生成器使用。    
  \item 符号表的内容:符号表通常包含变量、函数、类等符号的信息,如名称、类型、作用域等。
\end{itemize}

\subsection{中间代码生成}
以
gcc\ -fdump-tree-all-graph\  test.cpp\ -o\ test
gcc\ -fdump-rtl-all-graph\  test.cpp\ -o\ test
命令,我们生成了中间代码的多阶段输出,其中.dot文件即文件的控制流图。
对于tree阶段和rtl阶段,我们得到了不同层次的中间表示,每次优化pass都会生成一个新的.dot文件。
例如生成图如下\ref{fig:o}、\ref{fig:o2}所示,图1图2分别是rtl和tree的控制流图,图3图4分别是rtl和tree的-O2优化控制流图。
\begin{figure}[H]
  \centering
  \begin{minipage}[t]{0.48\textwidth}
    \centering
    \includegraphics[width=0.95\textwidth]{dot_r.png}
    \caption{图1}
    
  \end{minipage}%
  \hfill
  \begin{minipage}[t]{0.48\textwidth}
    \centering
    \includegraphics[width=0.95\textwidth]{dot_t.png}
    \caption{图2}
    
  \end{minipage}
  \label{fig:o}
\end{figure}

\begin{figure}[H]
  \centering
  \begin{minipage}[t]{0.48\textwidth}
    \centering
    \includegraphics[width=0.95\textwidth]{dot_r2.png}
    \caption{图3}
    
  \end{minipage}%
  \hfill
  \begin{minipage}[t]{0.48\textwidth}
    \centering
    \includegraphics[width=0.95\textwidth]{dot_t2.png}
    \caption{图4}
    
  \end{minipage}
  \label{fig:o2}
\end{figure}
\subsection{代码优化}
代码优化是编译器中的一个重要阶段,旨在提高生成代码的性能和效率。代码优化可以分为两类:机器无关优化和机器相关优化。
对于中间代码生成,利用clang -S -emit-llvm main.c指令,我们可以
通过-O1、-O2、-O3等不同的优化级别来生成不同优化程度的LLVM中间代码,优化后代码如下。

\begin{lstlisting}[title=O1优化,frame=trbl]
; ModuleID = 'test.cpp'
source_filename = "test.cpp"
target datalayout = "e-m:e-p270:32:32-p271:32:32-p272:64:64-i64:64-i128:128-f80:128-n8:16:32:64-S128"
target triple = "x86_64-pc-linux-gnu"

module asm ".globl _ZSt21ios_base_library_initv"

%"class.std::basic_istream" = type { ptr, i64, %"class.std::basic_ios" }
%"class.std::basic_ios" = type { %"class.std::ios_base", ptr, i8, i8, ptr, ptr, ptr, ptr }
%"class.std::ios_base" = type { ptr, i64, i64, i32, i32, i32, ptr, %"struct.std::ios_base::_Words", [8 x %"struct.std::ios_base::_Words"], i32, ptr, %"class.std::locale" }
%"struct.std::ios_base::_Words" = type { ptr, i64 }
%"class.std::locale" = type { ptr }
%"class.std::basic_ostream" = type { ptr, %"class.std::basic_ios" }
%"class.std::ctype" = type <{ %"class.std::locale::facet.base", [4 x i8], ptr, i8, [7 x i8], ptr, ptr, ptr, i8, [256 x i8], [256 x i8], i8, [6 x i8] }>
%"class.std::locale::facet.base" = type <{ ptr, i32 }>

@_ZSt3cin = external global %"class.std::basic_istream", align 8
@_ZSt4cout = external global %"class.std::basic_ostream", align 8

; Function Attrs: mustprogress norecurse uwtable
define dso_local noundef i32 @main() local_unnamed_addr #0 {
  %1 = alloca i32, align 4
  call void @llvm.lifetime.start.p0(i64 4, ptr nonnull %1) #4
  %2 = call noundef nonnull align 8 dereferenceable(16) ptr @_ZNSirsERi(ptr noundef nonnull align 8 dereferenceable(16) @_ZSt3cin, ptr noundef nonnull align 4 dereferenceable(4) %1)
  %3 = load i32, ptr %1, align 4, !tbaa !5
  %4 = icmp slt i32 %3, 2
  br i1 %4, label %11, label %5

5:                                                ; preds = %0, %5
  %6 = phi i32 [ %8, %5 ], [ 1, %0 ]
  %7 = phi i32 [ %9, %5 ], [ 2, %0 ]
  %8 = mul nsw i32 %6, %7
  %9 = add nuw i32 %7, 1
  %10 = icmp eq i32 %7, %3
  br i1 %10, label %11, label %5, !llvm.loop !9

11:                                               ; preds = %5, %0
  %12 = phi i32 [ 1, %0 ], [ %8, %5 ]
  %13 = call noundef nonnull align 8 dereferenceable(8) ptr @_ZNSolsEi(ptr noundef nonnull align 8 dereferenceable(8) @_ZSt4cout, i32 noundef %12)
  %14 = load ptr, ptr %13, align 8, !tbaa !12
  %15 = getelementptr i8, ptr %14, i64 -24
  %16 = load i64, ptr %15, align 8
  %17 = getelementptr inbounds i8, ptr %13, i64 %16
  %18 = getelementptr inbounds %"class.std::basic_ios", ptr %17, i64 0, i32 5
  %19 = load ptr, ptr %18, align 8, !tbaa !14
  %20 = icmp eq ptr %19, null
  br i1 %20, label %21, label %22

21:                                               ; preds = %11
  call void @_ZSt16__throw_bad_castv() #5
  unreachable

22:                                               ; preds = %11
  %23 = getelementptr inbounds %"class.std::ctype", ptr %19, i64 0, i32 8
  %24 = load i8, ptr %23, align 8, !tbaa !24
  %25 = icmp eq i8 %24, 0
  br i1 %25, label %29, label %26

26:                                               ; preds = %22
  %27 = getelementptr inbounds %"class.std::ctype", ptr %19, i64 0, i32 9, i64 10
  %28 = load i8, ptr %27, align 1, !tbaa !27
  br label %34

29:                                               ; preds = %22
  call void @_ZNKSt5ctypeIcE13_M_widen_initEv(ptr noundef nonnull align 8 dereferenceable(570) %19)
  %30 = load ptr, ptr %19, align 8, !tbaa !12
  %31 = getelementptr inbounds ptr, ptr %30, i64 6
  %32 = load ptr, ptr %31, align 8
  %33 = call noundef signext i8 %32(ptr noundef nonnull align 8 dereferenceable(570) %19, i8 noundef signext 10)
  br label %34

34:                                               ; preds = %26, %29
  %35 = phi i8 [ %28, %26 ], [ %33, %29 ]
  %36 = call noundef nonnull align 8 dereferenceable(8) ptr @_ZNSo3putEc(ptr noundef nonnull align 8 dereferenceable(8) %13, i8 noundef signext %35)
  %37 = call noundef nonnull align 8 dereferenceable(8) ptr @_ZNSo5flushEv(ptr noundef nonnull align 8 dereferenceable(8) %36)
  call void @llvm.lifetime.end.p0(i64 4, ptr nonnull %1) #4
  ret i32 0
}

; Function Attrs: mustprogress nocallback nofree nosync nounwind willreturn memory(argmem: readwrite)
declare void @llvm.lifetime.start.p0(i64 immarg, ptr nocapture) #1

declare noundef nonnull align 8 dereferenceable(16) ptr @_ZNSirsERi(ptr noundef nonnull align 8 dereferenceable(16), ptr noundef nonnull align 4 dereferenceable(4)) local_unnamed_addr #2

declare noundef nonnull align 8 dereferenceable(8) ptr @_ZNSolsEi(ptr noundef nonnull align 8 dereferenceable(8), i32 noundef) local_unnamed_addr #2

; Function Attrs: mustprogress nocallback nofree nosync nounwind willreturn memory(argmem: readwrite)
declare void @llvm.lifetime.end.p0(i64 immarg, ptr nocapture) #1

declare noundef nonnull align 8 dereferenceable(8) ptr @_ZNSo3putEc(ptr noundef nonnull align 8 dereferenceable(8), i8 noundef signext) local_unnamed_addr #2

declare noundef nonnull align 8 dereferenceable(8) ptr @_ZNSo5flushEv(ptr noundef nonnull align 8 dereferenceable(8)) local_unnamed_addr #2

; Function Attrs: noreturn
declare void @_ZSt16__throw_bad_castv() local_unnamed_addr #3

declare void @_ZNKSt5ctypeIcE13_M_widen_initEv(ptr noundef nonnull align 8 dereferenceable(570)) local_unnamed_addr #2

attributes #0 = { mustprogress norecurse uwtable "min-legal-vector-width"="0" "no-trapping-math"="true" "stack-protector-buffer-size"="8" "target-cpu"="x86-64" "target-features"="+cmov,+cx8,+fxsr,+mmx,+sse,+sse2,+x87" "tune-cpu"="generic" }
attributes #1 = { mustprogress nocallback nofree nosync nounwind willreturn memory(argmem: readwrite) }
attributes #2 = { "no-trapping-math"="true" "stack-protector-buffer-size"="8" "target-cpu"="x86-64" "target-features"="+cmov,+cx8,+fxsr,+mmx,+sse,+sse2,+x87" "tune-cpu"="generic" }
attributes #3 = { noreturn "no-trapping-math"="true" "stack-protector-buffer-size"="8" "target-cpu"="x86-64" "target-features"="+cmov,+cx8,+fxsr,+mmx,+sse,+sse2,+x87" "tune-cpu"="generic" }
attributes #4 = { nounwind }
attributes #5 = { noreturn }

!llvm.module.flags = !{!0, !1, !2, !3}
!llvm.ident = !{!4}

!0 = !{i32 1, !"wchar_size", i32 4}
!1 = !{i32 8, !"PIC Level", i32 2}
!2 = !{i32 7, !"PIE Level", i32 2}
!3 = !{i32 7, !"uwtable", i32 2}
!4 = !{!"Ubuntu clang version 18.1.3 (1ubuntu1)"}
!5 = !{!6, !6, i64 0}
!6 = !{!"int", !7, i64 0}
!7 = !{!"omnipotent char", !8, i64 0}
!8 = !{!"Simple C++ TBAA"}
!9 = distinct !{!9, !10, !11}
!10 = !{!"llvm.loop.mustprogress"}
!11 = !{!"llvm.loop.unroll.disable"}
!12 = !{!13, !13, i64 0}
!13 = !{!"vtable pointer", !8, i64 0}
!14 = !{!15, !20, i64 240}
!15 = !{!"_ZTSSt9basic_iosIcSt11char_traitsIcEE", !16, i64 0, !20, i64 216, !7, i64 224, !23, i64 225, !20, i64 232, !20, i64 240, !20, i64 248, !20, i64 256}
!16 = !{!"_ZTSSt8ios_base", !17, i64 8, !17, i64 16, !18, i64 24, !19, i64 28, !19, i64 32, !20, i64 40, !21, i64 48, !7, i64 64, !6, i64 192, !20, i64 200, !22, i64 208}
!17 = !{!"long", !7, i64 0}
!18 = !{!"_ZTSSt13_Ios_Fmtflags", !7, i64 0}
!19 = !{!"_ZTSSt12_Ios_Iostate", !7, i64 0}
!20 = !{!"any pointer", !7, i64 0}
!21 = !{!"_ZTSNSt8ios_base6_WordsE", !20, i64 0, !17, i64 8}
!22 = !{!"_ZTSSt6locale", !20, i64 0}
!23 = !{!"bool", !7, i64 0}
!24 = !{!25, !7, i64 56}
!25 = !{!"_ZTSSt5ctypeIcE", !26, i64 0, !20, i64 16, !23, i64 24, !20, i64 32, !20, i64 40, !20, i64 48, !7, i64 56, !7, i64 57, !7, i64 313, !7, i64 569}
!26 = !{!"_ZTSNSt6locale5facetE", !6, i64 8}
!27 = !{!7, !7, i64 0}

\end{lstlisting}


\begin{lstlisting}[title=O2优化,frame=trbl]
; ModuleID = 'test.cpp'
source_filename = "test.cpp"
target datalayout = "e-m:e-p270:32:32-p271:32:32-p272:64:64-i64:64-i128:128-f80:128-n8:16:32:64-S128"
target triple = "x86_64-pc-linux-gnu"

module asm ".globl _ZSt21ios_base_library_initv"

%"class.std::basic_istream" = type { ptr, i64, %"class.std::basic_ios" }
%"class.std::basic_ios" = type { %"class.std::ios_base", ptr, i8, i8, ptr, ptr, ptr, ptr }
%"class.std::ios_base" = type { ptr, i64, i64, i32, i32, i32, ptr, %"struct.std::ios_base::_Words", [8 x %"struct.std::ios_base::_Words"], i32, ptr, %"class.std::locale" }
%"struct.std::ios_base::_Words" = type { ptr, i64 }
%"class.std::locale" = type { ptr }
%"class.std::basic_ostream" = type { ptr, %"class.std::basic_ios" }
%"class.std::ctype" = type <{ %"class.std::locale::facet.base", [4 x i8], ptr, i8, [7 x i8], ptr, ptr, ptr, i8, [256 x i8], [256 x i8], i8, [6 x i8] }>
%"class.std::locale::facet.base" = type <{ ptr, i32 }>

@_ZSt3cin = external global %"class.std::basic_istream", align 8
@_ZSt4cout = external global %"class.std::basic_ostream", align 8

; Function Attrs: mustprogress norecurse uwtable
define dso_local noundef i32 @main() local_unnamed_addr #0 {
  %1 = alloca i32, align 4
  call void @llvm.lifetime.start.p0(i64 4, ptr nonnull %1) #5
  %2 = call noundef nonnull align 8 dereferenceable(16) ptr @_ZNSirsERi(ptr noundef nonnull align 8 dereferenceable(16) @_ZSt3cin, ptr noundef nonnull align 4 dereferenceable(4) %1)
  %3 = load i32, ptr %1, align 4, !tbaa !5
  %4 = icmp slt i32 %3, 2
  br i1 %4, label %35, label %5

5:                                                ; preds = %0
  %6 = add nsw i32 %3, -1
  %7 = icmp ult i32 %3, 9
  br i1 %7, label %26, label %8

8:                                                ; preds = %5
  %9 = and i32 %6, -8
  %10 = or disjoint i32 %9, 2
  br label %11

11:                                               ; preds = %11, %8
  %12 = phi i32 [ 0, %8 ], [ %19, %11 ]
  %13 = phi <4 x i32> [ <i32 1, i32 1, i32 1, i32 1>, %8 ], [ %17, %11 ]
  %14 = phi <4 x i32> [ <i32 1, i32 1, i32 1, i32 1>, %8 ], [ %18, %11 ]
  %15 = phi <4 x i32> [ <i32 2, i32 3, i32 4, i32 5>, %8 ], [ %20, %11 ]
  %16 = add <4 x i32> %15, <i32 4, i32 4, i32 4, i32 4>
  %17 = mul <4 x i32> %13, %15
  %18 = mul <4 x i32> %14, %16
  %19 = add nuw i32 %12, 8
  %20 = add <4 x i32> %15, <i32 8, i32 8, i32 8, i32 8>
  %21 = icmp eq i32 %19, %9
  br i1 %21, label %22, label %11, !llvm.loop !9

22:                                               ; preds = %11
  %23 = mul <4 x i32> %18, %17
  %24 = call i32 @llvm.vector.reduce.mul.v4i32(<4 x i32> %23)
  %25 = icmp eq i32 %6, %9
  br i1 %25, label %35, label %26

26:                                               ; preds = %5, %22
  %27 = phi i32 [ 1, %5 ], [ %24, %22 ]
  %28 = phi i32 [ 2, %5 ], [ %10, %22 ]
  br label %29

29:                                               ; preds = %26, %29
  %30 = phi i32 [ %32, %29 ], [ %27, %26 ]
  %31 = phi i32 [ %33, %29 ], [ %28, %26 ]
  %32 = mul nsw i32 %30, %31
  %33 = add nuw i32 %31, 1
  %34 = icmp eq i32 %31, %3
  br i1 %34, label %35, label %29, !llvm.loop !13

35:                                               ; preds = %29, %22, %0
  %36 = phi i32 [ 1, %0 ], [ %24, %22 ], [ %32, %29 ]
  %37 = call noundef nonnull align 8 dereferenceable(8) ptr @_ZNSolsEi(ptr noundef nonnull align 8 dereferenceable(8) @_ZSt4cout, i32 noundef %36)
  %38 = load ptr, ptr %37, align 8, !tbaa !14
  %39 = getelementptr i8, ptr %38, i64 -24
  %40 = load i64, ptr %39, align 8
  %41 = getelementptr inbounds i8, ptr %37, i64 %40
  %42 = getelementptr inbounds %"class.std::basic_ios", ptr %41, i64 0, i32 5
  %43 = load ptr, ptr %42, align 8, !tbaa !16
  %44 = icmp eq ptr %43, null
  br i1 %44, label %45, label %46

45:                                               ; preds = %35
  call void @_ZSt16__throw_bad_castv() #6
  unreachable

46:                                               ; preds = %35
  %47 = getelementptr inbounds %"class.std::ctype", ptr %43, i64 0, i32 8
  %48 = load i8, ptr %47, align 8, !tbaa !26
  %49 = icmp eq i8 %48, 0
  br i1 %49, label %53, label %50

50:                                               ; preds = %46
  %51 = getelementptr inbounds %"class.std::ctype", ptr %43, i64 0, i32 9, i64 10
  %52 = load i8, ptr %51, align 1, !tbaa !29
  br label %58

53:                                               ; preds = %46
  call void @_ZNKSt5ctypeIcE13_M_widen_initEv(ptr noundef nonnull align 8 dereferenceable(570) %43)
  %54 = load ptr, ptr %43, align 8, !tbaa !14
  %55 = getelementptr inbounds ptr, ptr %54, i64 6
  %56 = load ptr, ptr %55, align 8
  %57 = call noundef signext i8 %56(ptr noundef nonnull align 8 dereferenceable(570) %43, i8 noundef signext 10)
  br label %58

58:                                               ; preds = %50, %53
  %59 = phi i8 [ %52, %50 ], [ %57, %53 ]
  %60 = call noundef nonnull align 8 dereferenceable(8) ptr @_ZNSo3putEc(ptr noundef nonnull align 8 dereferenceable(8) %37, i8 noundef signext %59)
  %61 = call noundef nonnull align 8 dereferenceable(8) ptr @_ZNSo5flushEv(ptr noundef nonnull align 8 dereferenceable(8) %60)
  call void @llvm.lifetime.end.p0(i64 4, ptr nonnull %1) #5
  ret i32 0
}

; Function Attrs: mustprogress nocallback nofree nosync nounwind willreturn memory(argmem: readwrite)
declare void @llvm.lifetime.start.p0(i64 immarg, ptr nocapture) #1

declare noundef nonnull align 8 dereferenceable(16) ptr @_ZNSirsERi(ptr noundef nonnull align 8 dereferenceable(16), ptr noundef nonnull align 4 dereferenceable(4)) local_unnamed_addr #2

declare noundef nonnull align 8 dereferenceable(8) ptr @_ZNSolsEi(ptr noundef nonnull align 8 dereferenceable(8), i32 noundef) local_unnamed_addr #2

; Function Attrs: mustprogress nocallback nofree nosync nounwind willreturn memory(argmem: readwrite)
declare void @llvm.lifetime.end.p0(i64 immarg, ptr nocapture) #1

declare noundef nonnull align 8 dereferenceable(8) ptr @_ZNSo3putEc(ptr noundef nonnull align 8 dereferenceable(8), i8 noundef signext) local_unnamed_addr #2

declare noundef nonnull align 8 dereferenceable(8) ptr @_ZNSo5flushEv(ptr noundef nonnull align 8 dereferenceable(8)) local_unnamed_addr #2

; Function Attrs: noreturn
declare void @_ZSt16__throw_bad_castv() local_unnamed_addr #3

declare void @_ZNKSt5ctypeIcE13_M_widen_initEv(ptr noundef nonnull align 8 dereferenceable(570)) local_unnamed_addr #2

; Function Attrs: nocallback nofree nosync nounwind speculatable willreturn memory(none)
declare i32 @llvm.vector.reduce.mul.v4i32(<4 x i32>) #4

attributes #0 = { mustprogress norecurse uwtable "min-legal-vector-width"="0" "no-trapping-math"="true" "stack-protector-buffer-size"="8" "target-cpu"="x86-64" "target-features"="+cmov,+cx8,+fxsr,+mmx,+sse,+sse2,+x87" "tune-cpu"="generic" }
attributes #1 = { mustprogress nocallback nofree nosync nounwind willreturn memory(argmem: readwrite) }
attributes #2 = { "no-trapping-math"="true" "stack-protector-buffer-size"="8" "target-cpu"="x86-64" "target-features"="+cmov,+cx8,+fxsr,+mmx,+sse,+sse2,+x87" "tune-cpu"="generic" }
attributes #3 = { noreturn "no-trapping-math"="true" "stack-protector-buffer-size"="8" "target-cpu"="x86-64" "target-features"="+cmov,+cx8,+fxsr,+mmx,+sse,+sse2,+x87" "tune-cpu"="generic" }
attributes #4 = { nocallback nofree nosync nounwind speculatable willreturn memory(none) }
attributes #5 = { nounwind }
attributes #6 = { noreturn }

!llvm.module.flags = !{!0, !1, !2, !3}
!llvm.ident = !{!4}

!0 = !{i32 1, !"wchar_size", i32 4}
!1 = !{i32 8, !"PIC Level", i32 2}
!2 = !{i32 7, !"PIE Level", i32 2}
!3 = !{i32 7, !"uwtable", i32 2}
!4 = !{!"Ubuntu clang version 18.1.3 (1ubuntu1)"}
!5 = !{!6, !6, i64 0}
!6 = !{!"int", !7, i64 0}
!7 = !{!"omnipotent char", !8, i64 0}
!8 = !{!"Simple C++ TBAA"}
!9 = distinct !{!9, !10, !11, !12}
!10 = !{!"llvm.loop.mustprogress"}
!11 = !{!"llvm.loop.isvectorized", i32 1}
!12 = !{!"llvm.loop.unroll.runtime.disable"}
!13 = distinct !{!13, !10, !12, !11}
!14 = !{!15, !15, i64 0}
!15 = !{!"vtable pointer", !8, i64 0}
!16 = !{!17, !22, i64 240}
!17 = !{!"_ZTSSt9basic_iosIcSt11char_traitsIcEE", !18, i64 0, !22, i64 216, !7, i64 224, !25, i64 225, !22, i64 232, !22, i64 240, !22, i64 248, !22, i64 256}
!18 = !{!"_ZTSSt8ios_base", !19, i64 8, !19, i64 16, !20, i64 24, !21, i64 28, !21, i64 32, !22, i64 40, !23, i64 48, !7, i64 64, !6, i64 192, !22, i64 200, !24, i64 208}
!19 = !{!"long", !7, i64 0}
!20 = !{!"_ZTSSt13_Ios_Fmtflags", !7, i64 0}
!21 = !{!"_ZTSSt12_Ios_Iostate", !7, i64 0}
!22 = !{!"any pointer", !7, i64 0}
!23 = !{!"_ZTSNSt8ios_base6_WordsE", !22, i64 0, !19, i64 8}
!24 = !{!"_ZTSSt6locale", !22, i64 0}
!25 = !{!"bool", !7, i64 0}
!26 = !{!27, !7, i64 56}
!27 = !{!"_ZTSSt5ctypeIcE", !28, i64 0, !22, i64 16, !25, i64 24, !22, i64 32, !22, i64 40, !22, i64 48, !7, i64 56, !7, i64 57, !7, i64 313, !7, i64 569}
!28 = !{!"_ZTSNSt6locale5facetE", !6, i64 8}
!29 = !{!7, !7, i64 0}

\end{lstlisting}

我们通过diff工具发现-O3与-O2代码并无区别,但是—O2与-O1区别较大,我们试分析其主要区别:


\paragraph{1. 循环优化相关元数据差异}
\begin{itemize}
    \item 旧文件:
    \begin{verbatim}
    !11 = !{!"llvm.loop.unroll.disable"}
    \end{verbatim}
    表示禁止循环展开。

    \item 新文件:
    \begin{verbatim}
    !11 = !{!"llvm.loop.isvectorized", i32 1}
    !12 = !{!"llvm.loop.unroll.runtime.disable"}
    !13 = distinct !{!13, !10, !12, !11}
    \end{verbatim}
    表示循环已向量化,运行时循环展开被禁用,并使用 \texttt{distinct} 元数据保证节点唯一性。
\end{itemize}

\paragraph{2. 类型与 vtable 元数据差异}
\begin{itemize}
    \item 旧文件:
    \begin{verbatim}
    !13 = !{!"vtable pointer", !8, i64 0}
    !15 = !{!"_ZTSSt9basic_iosIcSt11char_traitsIcEE", ...}
    \end{verbatim}

    \item 新文件:
    \begin{verbatim}
    !15 = !{!"vtable pointer", !8, i64 0}
    !17 = !{!"_ZTSSt9basic_iosIcSt11char_traitsIcEE", ...}
    \end{verbatim}

    \item 说明:
    \begin{itemize}
        \item 描述 C++ 类型布局、vtable 偏移和成员偏移。
        \item 新文件调整了节点编号和顺序,但类型信息逻辑不变。
    \end{itemize}
\end{itemize}

\paragraph{3. 指针与基本类型元数据差异}
\begin{itemize}
    \item 旧文件:
    \begin{verbatim}
    !20 = !{!"any pointer", !7, i64 0}
    !21 = !{!"_ZTSNSt8ios_base6_WordsE", !20, i64 0, !17, i64 8}
    \end{verbatim}

    \item 新文件:
    \begin{verbatim}
    !22 = !{!"any pointer", !7, i64 0}
    !23 = !{!"_ZTSNSt8ios_base6_WordsE", !22, i64 0, !19, i64 8}
    \end{verbatim}

    \item 说明:
    \begin{itemize}
        \item 元数据表示指针类型和基本类型的偏移信息。
        \item 差异仅在节点编号或顺序,逻辑不变。
    \end{itemize}
\end{itemize}

\paragraph{4. 总结}
\begin{itemize}
    \item 两个 diff 的差异主要体现在:
    \begin{enumerate}
        \item 循环属性变化,如从 \texttt{unroll.disable} 到 \texttt{isvectorized}。
        \item metadata 节点编号和顺序调整。
        \item C++ 类型布局和 vtable 信息重新组织。
    \end{enumerate}
    \item 这些差异不会影响程序逻辑,仅用于 LLVM 优化器做:
    \begin{itemize}
        \item 循环优化决策
        \item 类型和别名分析
        \item 调试信息生成
    \end{itemize}
\end{itemize}
\par
对于opt指定使用某个pass优化.ll文件,我们发现其对于pass选择很重要,
我试了几个pass都未发现显著的差异,因此并不具体比较。

\subsection{代码生成}
我们通过gcc,clang+llc,和aarch64-linux-gnu-gc一共生成了三段汇编代码。
其中gcc生成的汇编代码与clang+llc生成的x86汇编代码几乎一致,
因此我们只展示其中两段代码,gcc其实就是clang+llc的一个封装。

\begin{lstlisting}[title=arm汇编,frame=trbl]
	.arch armv8-a
	.file	"test.cpp"
	.text
#APP
	.globl _ZSt21ios_base_library_initv
#NO_APP
	.align	2
	.global	main
	.type	main, %function
main:
.LFB1986:
	.cfi_startproc
	sub	sp, sp, #48
	.cfi_def_cfa_offset 48
	stp	x29, x30, [sp, 32]
	.cfi_offset 29, -16
	.cfi_offset 30, -8
	add	x29, sp, 32
	adrp	x0, :got:__stack_chk_guard
	ldr	x0, [x0, :got_lo12:__stack_chk_guard]
	ldr	x1, [x0]
	str	x1, [sp, 24]
	mov	x1, 0
	add	x0, sp, 12
	mov	x1, x0
	adrp	x0, :got:_ZSt3cin
	ldr	x0, [x0, :got_lo12:_ZSt3cin]
	bl	_ZNSirsERi
	mov	w0, 2
	str	w0, [sp, 16]
	mov	w0, 1
	str	w0, [sp, 20]
	b	.L2
.L3:
	ldr	w1, [sp, 20]
	ldr	w0, [sp, 16]
	mul	w0, w1, w0
	str	w0, [sp, 20]
	ldr	w0, [sp, 16]
	add	w0, w0, 1
	str	w0, [sp, 16]
.L2:
	ldr	w0, [sp, 12]
	ldr	w1, [sp, 16]
	cmp	w1, w0
	ble	.L3
	ldr	w1, [sp, 20]
	adrp	x0, :got:_ZSt4cout
	ldr	x0, [x0, :got_lo12:_ZSt4cout]
	bl	_ZNSolsEi
	adrp	x1, :got:_ZSt4endlIcSt11char_traitsIcEERSt13basic_ostreamIT_T0_ES6_
	ldr	x1, [x1, :got_lo12:_ZSt4endlIcSt11char_traitsIcEERSt13basic_ostreamIT_T0_ES6_]
	bl	_ZNSolsEPFRSoS_E
	mov	w0, 0
	mov	w1, w0
	adrp	x0, :got:__stack_chk_guard
	ldr	x0, [x0, :got_lo12:__stack_chk_guard]
	ldr	x3, [sp, 24]
	ldr	x2, [x0]
	subs	x3, x3, x2
	mov	x2, 0
	beq	.L5
	bl	__stack_chk_fail
.L5:
	mov	w0, w1
	ldp	x29, x30, [sp, 32]
	add	sp, sp, 48
	.cfi_restore 29
	.cfi_restore 30
	.cfi_def_cfa_offset 0
	ret
	.cfi_endproc
.LFE1986:
	.size	main, .-main
	.section	.rodata
	.type	_ZNSt8__detail30__integer_to_chars_is_unsignedIjEE, %object
	.size	_ZNSt8__detail30__integer_to_chars_is_unsignedIjEE, 1
_ZNSt8__detail30__integer_to_chars_is_unsignedIjEE:
	.byte	1
	.type	_ZNSt8__detail30__integer_to_chars_is_unsignedImEE, %object
	.size	_ZNSt8__detail30__integer_to_chars_is_unsignedImEE, 1
_ZNSt8__detail30__integer_to_chars_is_unsignedImEE:
	.byte	1
	.type	_ZNSt8__detail30__integer_to_chars_is_unsignedIyEE, %object
	.size	_ZNSt8__detail30__integer_to_chars_is_unsignedIyEE, 1
_ZNSt8__detail30__integer_to_chars_is_unsignedIyEE:
	.byte	1
	.ident	"GCC: (Ubuntu 13.3.0-6ubuntu2~24.04) 13.3.0"
	.section	.note.GNU-stack,"",@progbits

\end{lstlisting}

\begin{lstlisting}[title=x86汇编,frame=trbl]
	.file	"test.cpp"
	.text
#APP
	.globl _ZSt21ios_base_library_initv
#NO_APP
	.globl	main
	.type	main, @function
main:
.LFB1988:
	.cfi_startproc
	endbr64
	pushq	%rbp
	.cfi_def_cfa_offset 16
	.cfi_offset 6, -16
	movq	%rsp, %rbp
	.cfi_def_cfa_register 6
	subq	$32, %rsp
	movq	%fs:40, %rax
	movq	%rax, -8(%rbp)
	xorl	%eax, %eax
	leaq	-20(%rbp), %rax
	movq	%rax, %rsi
	leaq	_ZSt3cin(%rip), %rax
	movq	%rax, %rdi
	call	_ZNSirsERi@PLT
	movl	$2, -16(%rbp)
	movl	$1, -12(%rbp)
	jmp	.L2
.L3:
	movl	-12(%rbp), %eax
	imull	-16(%rbp), %eax
	movl	%eax, -12(%rbp)
	addl	$1, -16(%rbp)
.L2:
	movl	-20(%rbp), %eax
	cmpl	%eax, -16(%rbp)
	jle	.L3
	movl	-12(%rbp), %eax
	movl	%eax, %esi
	leaq	_ZSt4cout(%rip), %rax
	movq	%rax, %rdi
	call	_ZNSolsEi@PLT
	movq	_ZSt4endlIcSt11char_traitsIcEERSt13basic_ostreamIT_T0_ES6_@GOTPCREL(%rip), %rdx
	movq	%rdx, %rsi
	movq	%rax, %rdi
	call	_ZNSolsEPFRSoS_E@PLT
	movl	$0, %eax
	movq	-8(%rbp), %rdx
	subq	%fs:40, %rdx
	je	.L5
	call	__stack_chk_fail@PLT
.L5:
	leave
	.cfi_def_cfa 7, 8
	ret
	.cfi_endproc
.LFE1988:
	.size	main, .-main
	.section	.rodata
	.type	_ZNSt8__detail30__integer_to_chars_is_unsignedIjEE, @object
	.size	_ZNSt8__detail30__integer_to_chars_is_unsignedIjEE, 1
_ZNSt8__detail30__integer_to_chars_is_unsignedIjEE:
	.byte	1
	.type	_ZNSt8__detail30__integer_to_chars_is_unsignedImEE, @object
	.size	_ZNSt8__detail30__integer_to_chars_is_unsignedImEE, 1
_ZNSt8__detail30__integer_to_chars_is_unsignedImEE:
	.byte	1
	.type	_ZNSt8__detail30__integer_to_chars_is_unsignedIyEE, @object
	.size	_ZNSt8__detail30__integer_to_chars_is_unsignedIyEE, 1
_ZNSt8__detail30__integer_to_chars_is_unsignedIyEE:
	.byte	1
	.ident	"GCC: (Ubuntu 13.3.0-6ubuntu2~24.04) 13.3.0"
	.section	.note.GNU-stack,"",@progbits
	.section	.note.gnu.property,"a"
	.align 8
	.long	1f - 0f
	.long	4f - 1f
	.long	5
0:
	.string	"GNU"
1:
	.align 8
	.long	0xc0000002
	.long	3f - 2f
2:
	.long	0x3
3:
	.align 8
4:

\end{lstlisting}

\section{汇编器处理分析与编译流程对比}

\subsection{汇编器功能与处理结果分析}

汇编器(Assembler)是编译工具链中的重要组成部分,其主要功能是将高级语言或中间表示(IR)转换为目标架构的机器码。汇编器的处理结果通常表现为汇编代码文件(.s 文件),其内容包括:

\begin{itemize}
    \item \textbf{指令序列}:对应目标 CPU 指令集的操作码(opcode)和操作数,最终会生成机器码。
    \item \textbf{符号表和重定位信息}:标识全局变量、函数地址等,供链接器使用。
    \item \textbf{伪指令}:如栈帧设置、段声明、调试信息(.file、.section、.cfi 等)。
    \item \textbf{优化信息}(可选):部分优化后的 IR 对应的汇编指令可能带有循环展开、向量化等标记。
\end{itemize}

汇编器具体功能分析如下:

\begin{enumerate}
    \item \textbf{指令选择(Instruction Selection)}:将 IR 或中间代码映射为目标架构的指令。
    \item \textbf{寄存器分配(Register Allocation)}:决定 IR 中虚拟寄存器对应目标寄存器。
    \item \textbf{指令调度(Instruction Scheduling)}:优化指令顺序以减少流水线冲突。
    \item \textbf{生成符号和重定位信息}:生成全局变量、函数符号及其偏移,供链接器处理。
    \item \textbf{伪指令处理}:如栈帧、调试信息、段信息的生成。
\end{enumerate}

汇编器输出的汇编代码是最终生成机器码前的中间结果,对于调试和性能分析非常重要。

\subsection{GCC 与 Clang+LLC 汇编处理流程对比}

\begin{table}[h!]
\centering
% 使用 tabularx 让列根据总宽自动换行,避免 Overfull hbox
\begin{tabularx}{\textwidth}{|>{\centering\arraybackslash}X|>{\centering\arraybackslash}X|>{\centering\arraybackslash}X|>{\centering\arraybackslash}X|}
\hline
工具链 & 输入 & 处理阶段 & 输出 \\
\hline
GCC & C/C++ 源代码 & 前端解析 + 中间优化 + 汇编后端 & 汇编代码(.s) \\
\hline
Clang + LLC & C/C++ 源代码 → LLVM IR & Clang 前端生成 LLVM IR → LLVM 优化 → LLC 后端生成汇编 & 汇编代码(.s) \\
\hline
aarch64-linux-gnu-gcc & C/C++ 源代码 & 与 GCC 类似,但后端针对 ARM64 架构优化 & 汇编代码(.s,ARM64 指令) \\
\hline
\end{tabularx}
\caption{不同工具链的汇编生成对比}
\end{table}

\subsubsection{GCC 编译生成汇编}
GCC 直接从 C/C++ 源码生成汇编,前端完成语法分析、类型检查和中间优化,后端将中间表示转换为目标指令。GCC 汇编结果中包含完整的伪指令和调试信息。

\subsubsection{Clang + LLC 编译生成汇编}
Clang 将 C/C++ 源码生成 LLVM IR,然后使用 LLVM 的优化 passes 对 IR 进行各种优化(如循环展开、向量化、死代码消除),最后由 \texttt{llc} 将优化后的 IR 转换为目标架构汇编。此方法可显式控制优化阶段,适合研究 LLVM IR 对汇编的影响。

\subsubsection{aarch64-linux-gnu-gcc 编译生成汇编}
aarch64-linux-gnu-gcc 是交叉编译工具链,用于生成 ARM64 架构汇编。其流程与 GCC 类似,但指令选择、寄存器分配和调度针对 AArch64 架构进行优化,生成的汇编文件适合 ARM64 处理器直接使用。

\subsection{总结}
\begin{itemize}
    \item 汇编器主要负责将中间表示或高层语言映射到目标机器指令,生成符号和伪指令。
    \item GCC 集成了前端、优化和后端,生成汇编快速方便。
    \item Clang+LLC 将前端与后端解耦,可对 IR 做详细优化分析。
    \item aarch64-linux-gnu-gcc 是针对 ARM64 的 GCC 变体,适合交叉编译 ARM 平台程序。
\end{itemize}

\section{链接器}
链接器(Linker)是编译工具链中的最后一个阶段,负责将多个目标文件(.o 文件)和库文件链接成最终的可执行文件或共享库。链接器的主要功能包括: 
\begin{itemize}
    \item \textbf{符号解析(Symbol Resolution)}:链接器会解析目标文件中的符号引用,确定每个符号(如函数名、变量名)对应的内存地址。如果一个符号在多个目标文件中定义,链接器会根据链接规则选择一个定义。
    \item \textbf{地址重定位(Relocation)}:链接器会调整目标文件中的地址引用,使其指向正确的内存位置。由于每个目标文件可能被加载到不同的内存地址,链接器需要更新所有相关的地址引用。
    \item \textbf{合并节(Section Merging)}:链接器会将多个目标文件中的相同类型的节(如代码节、数据节)合并成一个节,以便生成最终的可执行文件。
    \item \textbf{生成可执行文件格式}:链接器会根据目标平台生成符合该平台要求的可执行文件格式,如 ELF(Linux)、PE(Windows)等。 
    \item \textbf{处理库文件}:链接器可以链接静态库(.a 文件)和动态库(.so 文件),将库中的符号和代码包含到最终的可执行文件中。
\end{itemize}
我们通过执行代码  g++\ test\_x86.o\ -o\ test\_x86   得到了最终的可执行文件。
我们通过反汇编的命令 objdump\ -d\ test\_x86  和  objdump\ -d\ test\_x86.o  分别得到了没有链接的时候和链接结束后的汇编代码。 
代码比较如下,可见链接前,
只有程序的本地代码,没有任何外部符号或库的引用。
主要是函数和控制流的实现。
程序中的所有操作都是静态的,指向常量或固定内存位置。
链接后,
包含了动态链接信息(如 .plt 和 .plt.got)。
调用标准库函数和外部符号会通过全局偏移表(GOT)进行解析。
在程序执行期间,动态链接器会解析这些符号并将它们与实际库中的符号链接起来。
\begin{lstlisting}[title=链接前后汇编代码对比,frame=trbl]
<链接前>
0000000000000000 <main>:
   0:   f3 0f 1e fa             endbr64
   4:   55                      push   %rbp
   5:   48 89 e5                mov    %rsp,%rbp
   8:   48 83 ec 20             sub    $0x20,%rsp
   c:   64 48 8b 04 25 28 00    mov    %fs:0x28,%rax
  13:   00 00 
  15:   48 89 45 f8             mov    %rax,-0x8(%rbp)
  19:   31 c0                   xor    %eax,%eax
  1b:   48 8d 45 ec             lea    -0x14(%rbp),%rax
  1f:   48 89 c6                mov    %rax,%rsi
  22:   48 8d 05 00 00 00 00    lea    0x0(%rip),%rax        # 29 <main+0x29>
  29:   48 89 c7                mov    %rax,%rdi
  2c:   e8 00 00 00 00          call   31 <main+0x31>
  31:   c7 45 f0 02 00 00 00    movl   $0x2,-0x10(%rbp)
  38:   c7 45 f4 01 00 00 00    movl   $0x1,-0xc(%rbp)
  3f:   eb 0e                   jmp    4f <main+0x4f>
  41:   8b 45 f4                mov    -0xc(%rbp),%eax
  44:   0f af 45 f0             imul   -0x10(%rbp),%eax
  48:   89 45 f4                mov    %eax,-0xc(%rbp)
  4b:   83 45 f0 01             addl   $0x1,-0x10(%rbp)
  4f:   8b 45 ec                mov    -0x14(%rbp),%eax
  52:   39 45 f0                cmp    %eax,-0x10(%rbp)
  55:   7e ea                   jle    41 <main+0x41>
  57:   8b 45 f4                mov    -0xc(%rbp),%eax
  5a:   89 c6                   mov    %eax,%esi
  5c:   48 8d 05 00 00 00 00    lea    0x0(%rip),%rax        # 63 <main+0x63>
  63:   48 89 c7                mov    %rax,%rdi
  66:   e8 00 00 00 00          call   6b <main+0x6b>
  6b:   48 8b 15 00 00 00 00    mov    0x0(%rip),%rdx        # 72 <main+0x72>
  72:   48 89 d6                mov    %rdx,%rsi
  75:   48 89 c7                mov    %rax,%rdi
  78:   e8 00 00 00 00          call   7d <main+0x7d>
  7d:   b8 00 00 00 00          mov    $0x0,%eax
  82:   48 8b 55 f8             mov    -0x8(%rbp),%rdx
  86:   64 48 2b 14 25 28 00    sub    %fs:0x28,%rdx
  8d:   00 00 
  8f:   74 05                   je     96 <main+0x96>
  91:   e8 00 00 00 00          call   96 <main+0x96>
  96:   c9                      leave
  97:   c3                      ret
  <链接后>
0000000000001000 <_init>:
    1000:       f3 0f 1e fa             endbr64
    1004:       48 83 ec 08             sub    $0x8,%rsp
    1008:       48 8b 05 e1 2f 00 00    mov    0x2fe1(%rip),%rax        # 3ff0 <__gmon_start__@Base>
    100f:       48 85 c0                test   %rax,%rax
    1012:       74 02                   je     1016 <_init+0x16>
    1014:       ff d0                   call   *%rax
    1016:       48 83 c4 08             add    $0x8,%rsp
    101a:       c3                      ret

Disassembly of section .plt:

0000000000001020 <.plt>:
    1020:       ff 35 7a 2f 00 00       push   0x2f7a(%rip)        # 3fa0 <_GLOBAL_OFFSET_TABLE_+0x8>
    1026:       ff 25 7c 2f 00 00       jmp    *0x2f7c(%rip)        # 3fa8 <_GLOBAL_OFFSET_TABLE_+0x10>
    102c:       0f 1f 40 00             nopl   0x0(%rax)
    1030:       f3 0f 1e fa             endbr64
    1034:       68 00 00 00 00          push   $0x0
    1039:       e9 e2 ff ff ff          jmp    1020 <_init+0x20>
    103e:       66 90                   xchg   %ax,%ax
    1040:       f3 0f 1e fa             endbr64
    1044:       68 01 00 00 00          push   $0x1
    1049:       e9 d2 ff ff ff          jmp    1020 <_init+0x20>
    104e:       66 90                   xchg   %ax,%ax
    1050:       f3 0f 1e fa             endbr64
    1054:       68 02 00 00 00          push   $0x2
    1059:       e9 c2 ff ff ff          jmp    1020 <_init+0x20>
    105e:       66 90                   xchg   %ax,%ax
    1060:       f3 0f 1e fa             endbr64
    1064:       68 03 00 00 00          push   $0x3
    1069:       e9 b2 ff ff ff          jmp    1020 <_init+0x20>
    106e:       66 90                   xchg   %ax,%ax

Disassembly of section .plt.got:

0000000000001070 <__cxa_finalize@plt>:
    1070:       f3 0f 1e fa             endbr64
    1074:       ff 25 56 2f 00 00       jmp    *0x2f56(%rip)        # 3fd0 <__cxa_finalize@GLIBC_2.2.5>
    107a:       66 0f 1f 44 00 00       nopw   0x0(%rax,%rax,1)

Disassembly of section .plt.sec:

0000000000001080 <_ZNSirsERi@plt>:
    1080:       f3 0f 1e fa             endbr64
    1084:       ff 25 26 2f 00 00       jmp    *0x2f26(%rip)        # 3fb0 <_ZNSirsERi@GLIBCXX_3.4>
    108a:       66 0f 1f 44 00 00       nopw   0x0(%rax,%rax,1)

0000000000001090 <_ZNSolsEPFRSoS_E@plt>:
    1090:       f3 0f 1e fa             endbr64
    1094:       ff 25 1e 2f 00 00       jmp    *0x2f1e(%rip)        # 3fb8 <_ZNSolsEPFRSoS_E@GLIBCXX_3.4>
    109a:       66 0f 1f 44 00 00       nopw   0x0(%rax,%rax,1)

00000000000010a0 <__stack_chk_fail@plt>:
    10a0:       f3 0f 1e fa             endbr64
    10a4:       ff 25 16 2f 00 00       jmp    *0x2f16(%rip)        # 3fc0 <__stack_chk_fail@GLIBC_2.4>
    10aa:       66 0f 1f 44 00 00       nopw   0x0(%rax,%rax,1)

00000000000010b0 <_ZNSolsEi@plt>:
    10b0:       f3 0f 1e fa             endbr64
    10b4:       ff 25 0e 2f 00 00       jmp    *0x2f0e(%rip)        # 3fc8 <_ZNSolsEi@GLIBCXX_3.4>
    10ba:       66 0f 1f 44 00 00       nopw   0x0(%rax,%rax,1)

Disassembly of section .text:

00000000000010c0 <_start>:
    10c0:       f3 0f 1e fa             endbr64
    10c4:       31 ed                   xor    %ebp,%ebp
    10c6:       49 89 d1                mov    %rdx,%r9
    10c9:       5e                      pop    %rsi
    10ca:       48 89 e2                mov    %rsp,%rdx
    10cd:       48 83 e4 f0             and    $0xfffffffffffffff0,%rsp
    10d1:       50                      push   %rax
    10d2:       54                      push   %rsp
    10d3:       45 31 c0                xor    %r8d,%r8d
    10d6:       31 c9                   xor    %ecx,%ecx
    10d8:       48 8d 3d ca 00 00 00    lea    0xca(%rip),%rdi        # 11a9 <main>
    10df:       ff 15 fb 2e 00 00       call   *0x2efb(%rip)        # 3fe0 <__libc_start_main@GLIBC_2.34>
    10e5:       f4                      hlt
    10e6:       66 2e 0f 1f 84 00 00    cs nopw 0x0(%rax,%rax,1)
    10ed:       00 00 00 

00000000000010f0 <deregister_tm_clones>:
    10f0:       48 8d 3d 19 2f 00 00    lea    0x2f19(%rip),%rdi        # 4010 <__TMC_END__>
    10f7:       48 8d 05 12 2f 00 00    lea    0x2f12(%rip),%rax        # 4010 <__TMC_END__>
    10fe:       48 39 f8                cmp    %rdi,%rax
    1101:       74 15                   je     1118 <deregister_tm_clones+0x28>
    1103:       48 8b 05 de 2e 00 00    mov    0x2ede(%rip),%rax        # 3fe8 <_ITM_deregisterTMCloneTable@Base>
    110a:       48 85 c0                test   %rax,%rax
    110d:       74 09                   je     1118 <deregister_tm_clones+0x28>
    110f:       ff e0                   jmp    *%rax
    1111:       0f 1f 80 00 00 00 00    nopl   0x0(%rax)
    1118:       c3                      ret
    1119:       0f 1f 80 00 00 00 00    nopl   0x0(%rax)

0000000000001120 <register_tm_clones>:
    1120:       48 8d 3d e9 2e 00 00    lea    0x2ee9(%rip),%rdi        # 4010 <__TMC_END__>
    1127:       48 8d 35 e2 2e 00 00    lea    0x2ee2(%rip),%rsi        # 4010 <__TMC_END__>
    112e:       48 29 fe                sub    %rdi,%rsi
    1131:       48 89 f0                mov    %rsi,%rax
    1134:       48 c1 ee 3f             shr    $0x3f,%rsi
    1138:       48 c1 f8 03             sar    $0x3,%rax
    113c:       48 01 c6                add    %rax,%rsi
    113f:       48 d1 fe                sar    $1,%rsi
    1142:       74 14                   je     1158 <register_tm_clones+0x38>
    1144:       48 8b 05 ad 2e 00 00    mov    0x2ead(%rip),%rax        # 3ff8 <_ITM_registerTMCloneTable@Base>
    114b:       48 85 c0                test   %rax,%rax
    114e:       74 08                   je     1158 <register_tm_clones+0x38>
    1150:       ff e0                   jmp    *%rax
    1152:       66 0f 1f 44 00 00       nopw   0x0(%rax,%rax,1)
    1158:       c3                      ret
    1159:       0f 1f 80 00 00 00 00    nopl   0x0(%rax)

0000000000001160 <__do_global_dtors_aux>:
    1160:       f3 0f 1e fa             endbr64
    1164:       80 3d 0d 31 00 00 00    cmpb   $0x0,0x310d(%rip)        # 4278 <completed.0>
    116b:       75 2b                   jne    1198 <__do_global_dtors_aux+0x38>
    116d:       55                      push   %rbp
    116e:       48 83 3d 5a 2e 00 00    cmpq   $0x0,0x2e5a(%rip)        # 3fd0 <__cxa_finalize@GLIBC_2.2.5>
    1175:       00 
    1176:       48 89 e5                mov    %rsp,%rbp
    1179:       74 0c                   je     1187 <__do_global_dtors_aux+0x27>
    117b:       48 8b 3d 86 2e 00 00    mov    0x2e86(%rip),%rdi        # 4008 <__dso_handle>
    1182:       e8 e9 fe ff ff          call   1070 <__cxa_finalize@plt>
    1187:       e8 64 ff ff ff          call   10f0 <deregister_tm_clones>
    118c:       c6 05 e5 30 00 00 01    movb   $0x1,0x30e5(%rip)        # 4278 <completed.0>
    1193:       5d                      pop    %rbp
    1194:       c3                      ret
    1195:       0f 1f 00                nopl   (%rax)
    1198:       c3                      ret
    1199:       0f 1f 80 00 00 00 00    nopl   0x0(%rax)

00000000000011a0 <frame_dummy>:
    11a0:       f3 0f 1e fa             endbr64
    11a4:       e9 77 ff ff ff          jmp    1120 <register_tm_clones>

00000000000011a9 <main>:
    11a9:       f3 0f 1e fa             endbr64
    11ad:       55                      push   %rbp
    11ae:       48 89 e5                mov    %rsp,%rbp
    11b1:       48 83 ec 20             sub    $0x20,%rsp
    11b5:       64 48 8b 04 25 28 00    mov    %fs:0x28,%rax
    11bc:       00 00 
    11be:       48 89 45 f8             mov    %rax,-0x8(%rbp)
    11c2:       31 c0                   xor    %eax,%eax
    11c4:       48 8d 45 ec             lea    -0x14(%rbp),%rax
    11c8:       48 89 c6                mov    %rax,%rsi
    11cb:       48 8d 05 8e 2f 00 00    lea    0x2f8e(%rip),%rax        # 4160 <_ZSt3cin@GLIBCXX_3.4>
    11d2:       48 89 c7                mov    %rax,%rdi
    11d5:       e8 a6 fe ff ff          call   1080 <_ZNSirsERi@plt>
    11da:       c7 45 f0 02 00 00 00    movl   $0x2,-0x10(%rbp)
    11e1:       c7 45 f4 01 00 00 00    movl   $0x1,-0xc(%rbp)
    11e8:       eb 0e                   jmp    11f8 <main+0x4f>
    11ea:       8b 45 f4                mov    -0xc(%rbp),%eax
    11ed:       0f af 45 f0             imul   -0x10(%rbp),%eax
    11f1:       89 45 f4                mov    %eax,-0xc(%rbp)
    11f4:       83 45 f0 01             addl   $0x1,-0x10(%rbp)
    11f8:       8b 45 ec                mov    -0x14(%rbp),%eax
    11fb:       39 45 f0                cmp    %eax,-0x10(%rbp)
    11fe:       7e ea                   jle    11ea <main+0x41>
    1200:       8b 45 f4                mov    -0xc(%rbp),%eax
    1203:       89 c6                   mov    %eax,%esi
    1205:       48 8d 05 34 2e 00 00    lea    0x2e34(%rip),%rax        # 4040 <_ZSt4cout@GLIBCXX_3.4>
    120c:       48 89 c7                mov    %rax,%rdi
    120f:       e8 9c fe ff ff          call   10b0 <_ZNSolsEi@plt>
    1214:       48 8b 15 bd 2d 00 00    mov    0x2dbd(%rip),%rdx        # 3fd8 <_ZSt4endlIcSt11char_traitsIcEERSt13basic_ostreamIT_T0_ES6_@GLIBCXX_3.4>
    121b:       48 89 d6                mov    %rdx,%rsi
    121e:       48 89 c7                mov    %rax,%rdi
    1221:       e8 6a fe ff ff          call   1090 <_ZNSolsEPFRSoS_E@plt>
    1226:       b8 00 00 00 00          mov    $0x0,%eax
    122b:       48 8b 55 f8             mov    -0x8(%rbp),%rdx
    122f:       64 48 2b 14 25 28 00    sub    %fs:0x28,%rdx
    1236:       00 00 
    1238:       74 05                   je     123f <main+0x96>
    123a:       e8 61 fe ff ff          call   10a0 <__stack_chk_fail@plt>
    123f:       c9                      leave
    1240:       c3                      ret

Disassembly of section .fini:

0000000000001244 <_fini>:
    1244:       f3 0f 1e fa             endbr64
    1248:       48 83 ec 08             sub    $0x8,%rsp
    124c:       48 83 c4 08             add    $0x8,%rsp
    1250:       c3                      ret
\end{lstlisting}





\newpage
\bibliographystyle{plain}
\bibliography{references} 
\end{document}
%——————————————————————————————————————
% \subsection{第一节}
% 如图\ref{fig:1}所示
% \begin{figure}[H]
%     \centering
%     \includegraphics[scale=0.3]{NKU.png}
%     \caption{Caption}
%     \label{fig:1}
% \end{figure}

% 表
% \begin{table}[!htbp]
%   \centering
%   \begin{tabular}{ccccccccccc}
%   \toprule  
%   N/n$\backslash$Algo& naive-conv& naive-pool& omp-conv& omp-pool\\
%   \midrule
%   64/2& 0.0167& 0.01255& 0.04142& 0.03799\\
%   64/4& 0.03599&0.0394& 0.0458& 0.0421\\
%   \bottomrule
%   \end{tabular}
%   \caption{性能测试结果(4线程)(单位:ms)}
% \end{table}

% 带单元格表格
% \begin{table}[!htbp]
%   \centering
%   \begin{tabular}{|c|c|c|c|c|c|c|}
%   \hline
%   \multicolumn{2}{|c|}{ \multirow{2}*{$Cost$} }& \multicolumn{5}{c|}{To}\\
%   \cline{3-7}
%   \multicolumn{2}{|c|}{}&$A$&$B$&$C$&$D$&$E$\\
%   \hline
%   \multirow{3}*{From}&$B$&7&0&1&3&8\\
%   \cline{2-7}
%   &$C$&8&1&0&2&7\\
%   \cline{2-7}
%   &$D$&8&3&2&0&5\\
%   \hline
%   \end{tabular}
%   \caption{结点C距离向量表(无毒性逆转)}
% \end{table}

% %——————————————————————————————————————
% \subsection{第二节}
% 伪代码

% \begin{breakablealgorithm} 
%   \caption{初始化obj文件信息——对应MeshSimplify类中readfile函数,Face类calMatrix函数} 
%   \begin{algorithmic}[1] %每行显示行号  
%       \Require obj文件,顶点、边、面列表
%       \Ensure 是否读取成功
%       \Function {calMatrix}{$Face$}  
%               \State $normal \gets e1×e2$  
%               \State $normal \gets normal/normal.length$
%               \State $temp[] \gets {normal.x, normal.y, normal.z, normal· Face.v1}$
%               \State $Matrix[i][j]=temp[i] * temp[j]$ 
%               \State \Return{$Matrix$}  
%       \EndFunction
%       \State 根据obj的v和f区分点面信息,读取并加入列表
%       \State $scale \gets $记录点坐标中距离原点最远的分量,以便后续OpenGL进行显示
%       \State $ori \gets $记录中心点,便于OpenGL显示在中心位置,避免有的obj偏移原点较多
%       \State 根据三角面片信息,计算一个面的三条边
%       \State 计算每个面的矩阵$\gets calMatrix$
%       \State 将每个面的矩阵加到各点,由点维护\\
%       \Return True
%   \end{algorithmic}  
% \end{breakablealgorithm}

% 代码
% \begin{lstlisting}[title=逐列访问平凡算法,frame=trbl,language={C++}]
%   void ord()   
%   {
%       double head,tail,freq,head1,tail1,timess=0; // timers
%       init(N);
%       QueryPerformanceFrequency((LARGE_INTEGER *)&freq );
%       QueryPerformanceCounter((LARGE_INTEGER *)&head);
%       for (int i=0; i<NN; i++)
%           for (int j=0; j<NN; j++)
%               col_sum[i] += (b[j][i]*a[j]);
%       QueryPerformanceCounter ((LARGE_INTEGER *)& tail) ;
%       cout << "\nordCol:" <<(tail-head)*1000.0 / freq<< "ms" << endl;
%   }
% \end{lstlisting}


% %——————————————————————————————————————
% \subsection{第三节}

% 参考文献\cite{adams1995hitchhiker}\cite{shin2016deep}
    
% 多行公式
% \begin{align}
%   a+b = a + b \\
%   \frac{a+b}{a-b}
% \end{align}

% 行内公式:$\sum^N_{i=1}$

% \textbf{超链接}  \href{http://youtube.com/}{YouTube}

% 带标号枚举
% \begin{enumerate}
%   \item 1
%   \item 2
% \end{enumerate}

% 不带标号枚举
% \begin{itemize}
%   \item 1
%   \item 2
% \end{itemize}

% \xiaosi{切换字体大小}

% %----------------------------------------------------------------
% \section{总结}

% %----------------------------------------------------------------

