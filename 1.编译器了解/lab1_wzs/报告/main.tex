\documentclass[a4paper]{article}

\input{style/ch_xelatex.tex}
\input{style/scala.tex}

%代码段设置
\lstset{numbers=left,
basicstyle=\tiny,
numberstyle=\tiny,
keywordstyle=\color{blue!70},
commentstyle=\color{red!50!green!50!blue!50},
frame=single, rulesepcolor=\color{red!20!green!20!blue!20},
escapeinside=``
}

\graphicspath{ {images/} }
\usepackage{ctex}
\usepackage{graphicx}
\usepackage{color,framed}%文本框
\usepackage{listings}
\usepackage{caption}
\usepackage{amssymb}
\usepackage{enumerate}
\usepackage{xcolor}
\usepackage{bm} 
\usepackage{lastpage}%获得总页数
\usepackage{fancyhdr}
\usepackage{tabularx}  
\usepackage{geometry}
%\usepackage{minted}  % removed to avoid requiring shell-escape and speed up compilation
\usepackage{graphics}
\usepackage{subfigure}
\usepackage{float}
\usepackage{pdfpages}
\usepackage{pgfplots}
\pgfplotsset{width=10cm,compat=1.9}
\usepackage{multirow}
\usepackage{footnote}
\usepackage{booktabs}
\lstdefinelanguage{text}{}
%-----------------------伪代码------------------
\usepackage{algorithm}  
\usepackage{algorithmicx}  
\usepackage{algpseudocode}  
\floatname{algorithm}{Algorithm}  
\renewcommand{\algorithmicrequire}{\textbf{Input:}}  
\renewcommand{\algorithmicensure}{\textbf{Output:}} 
\usepackage{lipsum}  
\makeatletter
\newenvironment{breakablealgorithm}
  {% \begin{breakablealgorithm}
  \begin{center}
     \refstepcounter{algorithm}% New algorithm
     \hrule height.8pt depth0pt \kern2pt% \@fs@pre for \@fs@ruled
     \renewcommand{\caption}[2][\relax]{% Make a new \caption
      {\raggedright\textbf{\ALG@name~\thealgorithm} ##2\par}%
      \ifx\relax##1\relax % #1 is \relax
         \addcontentsline{loa}{algorithm}{\protect\numberline{\thealgorithm}##2}%
      \else % #1 is not \relax
         \addcontentsline{loa}{algorithm}{\protect\numberline{\thealgorithm}##1}%
      \fi
      \kern2pt\hrule\kern2pt
     }
  }{% \end{breakablealgorithm}
     \kern2pt\hrule\relax% \@fs@post for \@fs@ruled
  \end{center}
  }
\makeatother
%------------------------代码-------------------
\usepackage{xcolor} 
\usepackage{listings} 
\lstset{ 
breaklines,%自动换行
basicstyle=\small,
escapeinside=``,
keywordstyle=\color{ blue!70} \bfseries,
commentstyle=\color{red!50!green!50!blue!50},% 
stringstyle=\ttfamily,% 
extendedchars=false,% 
linewidth=\textwidth,% 
numbers=left,% 
numberstyle=\tiny \color{blue!50},% 
frame=trbl% 
rulesepcolor= \color{ red!20!green!20!blue!20} 
}

%-------------------------页面边距--------------
\geometry{a4paper,left=2.3cm,right=2.3cm,top=2.7cm,bottom=2.7cm}
%-------------------------页眉页脚--------------
\usepackage{fancyhdr}
\pagestyle{fancy}
\lhead{\kaishu \leftmark}
% \chead{}
\rhead{\kaishu 编译原理实验报告}%加粗\bfseries 
\lfoot{}
\cfoot{\thepage}
\rfoot{}
\renewcommand{\headrulewidth}{0.1pt}  
\renewcommand{\footrulewidth}{0pt}%去掉横线
\newcommand{\HRule}{\rule{\linewidth}{0.5mm}}%标题横线
\newcommand{\HRulegrossa}{\rule{\linewidth}{1.2mm}}
\setlength{\textfloatsep}{10mm}%设置图片的前后间距
%--------------------文档内容--------------------

\begin{document}
\renewcommand{\contentsname}{目\ 录}
\renewcommand{\appendixname}{附录}
\renewcommand{\appendixpagename}{附录}
\renewcommand{\refname}{参考文献} 
\renewcommand{\figurename}{图}
\renewcommand{\tablename}{表}
\renewcommand{\today}{\number\year 年 \number\month 月 \number\day 日}

%-------------------------封面----------------
\begin{titlepage}
    \begin{center}
    \includegraphics[width=0.8\textwidth]{NKU.png}\\[1cm]
    \vspace{20mm}
		\textbf{\huge\textbf{\kaishu{计算机学院}}}\\[0.5cm]
		\textbf{\huge{\kaishu{编译原理第一次试验}}}\\[2.3cm]
		\textbf{\Huge\textbf{\kaishu{认识编译器}}}

		\vspace{\fill}
    
    % \textbf{\Large \textbf{并行程序设计期末实验报告}}\\[0.8cm]
    % \HRule \\[0.9cm]
    % \HRule \\[2.0cm]
    \centering
    \textsc{\LARGE \kaishu{姓名\ :\ 王泽舜\ 杨乔钦}}\\[0.5cm]
    \textsc{\LARGE \kaishu{学号\ :\ 2310655\ 2311838}}\\[0.5cm]
    \textsc{\LARGE \kaishu{专业\ :\ 计算机科学与技术}}\\[0.5cm]
    
    \vfill
    {\Large \today}
    \end{center}
\end{titlepage}

\renewcommand {\thefigure}{\thesection{}.\arabic{figure}}%图片按章标号
\renewcommand{\figurename}{图}
\renewcommand{\contentsname}{目录}  
\cfoot{\thepage\ of \pageref{LastPage}}%当前页 of 总页数


% 生成目录
\clearpage
\tableofcontents
\newpage



%-------------------------Codes----------------------------------


\section{第一题:编译过程分析}

\subsection{预处理器}

预处理器是编译过程的第一阶段,主要负责处理源代码中的预处理指令,如宏定义、文件包含、条件编译等。在GCC中,预处理器由cpp工具执行。

以fibonacci.c为例,源代码包含了\#include <stdio.h>和\#define MAX\_ITERATIONS 10。预处理器会展开这些指令:将stdio.h的内容插入到代码中,替换宏定义等。生成的fibonacci.i文件就是预处理后的中间文件,其中包含了所有必要的头文件定义和宏展开,但不包含注释和预处理指令。

对比fibonacci.c和fibonacci.i:fibonacci.c中只有几行代码,而fibonacci.i包含了stdio.h的完整定义,包括函数声明如printf、putchar等。宏MAX\_ITERATIONS被替换为10。预处理器还添加了行号信息和编译器版本等元数据。

\textbf{之前(fibonacci.c的main函数部分):}
\begin{lstlisting}[language=C]
#define MAX_ITERATIONS 10

int main() {
    int a = 0, b = 1, next, i;

    printf("Fibonacci Series up to %d terms:\n", MAX_ITERATIONS);
    
    for (i = 0; i < MAX_ITERATIONS; i++) {
        if (i <= 1) {
            next = i;
        } else {
            next = a + b;
            a = b;
            b = next;
        }
        printf("%d ", next);
    }
    printf("\n");
    return 0;
}
\end{lstlisting}

\textbf{之后(fibonacci.i的main函数部分):}
\begin{lstlisting}[language=C]
int main() {
    int a = 0, b = 1, next, i;

    printf("Fibonacci Series up to %d terms:\n", 10);
    
    for (i = 0; i < 10; i++) {
        if (i <= 1) {
            next = i;
        } else {
            next = a + b;
            a = b;
            b = next;
        }
        printf("%d ", next);
    }
    printf("\n");
    return 0;
}
\end{lstlisting}

\subsection{编译器}

编译器将预处理后的代码转换为汇编代码。在GCC中,编译器前端(cc1)执行以下阶段:

词法分析:将源代码分解成token,如关键字(int, for, if)、标识符(main, a, b)、运算符(+, =)、常量(10)等。

语法分析:根据语法规则构建抽象语法树(AST),识别程序结构如函数定义、循环、条件语句。

语义分析:检查类型一致性、作用域等,进行类型检查和符号表构建。例如,确保变量类型匹配,函数调用正确。

中间代码生成:生成中间表示(IR),如GIMPLE或RTL。GCC使用GIMPLE作为高级IR,然后转换为RTL。

优化:进行各种优化,如常量折叠(将MAX\_ITERATIONS替换为10)、死代码消除、循环优化等。在fibonacci程序中,循环被优化为汇编中的jmp和cmp指令。

代码生成:将IR转换为目标平台的汇编代码。

从fibonacci.i到fibonacci.s,编译器生成了x86-64汇编代码。fibonacci.s包含了.text段、.rodata段(字符串常量)、main函数的汇编实现。汇编代码使用了栈来管理局部变量(如a, b, next, i),并调用printf和putchar。相比.i文件,.s文件是平台特定的机器指令序列。

\textbf{之前(fibonacci.i的main函数部分):}
\begin{lstlisting}[language=C]
int main() {
    int a = 0, b = 1, next, i;

    printf("Fibonacci Series up to %d terms:\n", 10);
    
    for (i = 0; i < 10; i++) {
        if (i <= 1) {
            next = i;
        } else {
            next = a + b;
            a = b;
            b = next;
        }
        printf("%d ", next);
    }
    printf("\n");
    return 0;
}
\end{lstlisting}

\textbf{之后(fibonacci.s的main函数部分):}
\begin{lstlisting}[language={[x86masm]Assembler}]
	.globl	main
	.type	main, @function
main:
.LFB0:
	.cfi_startproc
	endbr64
	pushq	%rbp
	.cfi_def_cfa_offset 16
	.cfi_offset 6, -16
	movq	%rsp, %rbp
	.cfi_def_cfa_register 6
	subq	$16, %rsp
	movl	$0, -16(%rbp)
	movl	$1, -12(%rbp)
	movl	$10, %esi
	leaq	.LC0(%rip), %rax
	movq	%rax, %rdi
	movl	$0, %eax
	call	printf@PLT
	movl	$0, -4(%rbp)
	jmp	.L2
.L2:
\end{lstlisting}

\subsection{汇编器}

汇编器将汇编代码转换为机器码,生成目标文件(.o文件)。在GCC中,由as工具执行。

fibonacci.s文件包含了汇编指令,如mov, add, call等。汇编器将其翻译成二进制机器码,并组织成ELF格式的目标文件,包括代码段(.text)、数据段(.rodata)等。fibonacci.o.asm是目标文件的反汇编,显示了机器码(如0xf3 0f 1e fa对应endbr64)和对应的汇编指令。目标文件包含了符号表,未解析的外部符号(如printf, putchar)标记为未定义,需要链接器解决。

对比fibonacci.s和fibonacci.o.asm:.s是文本汇编,.o.asm是二进制反汇编,显示了实际的十六进制机器码和地址。

\textbf{之前(fibonacci.s的main函数部分):}
\begin{lstlisting}[language={[x86masm]Assembler}]
	.globl	main
	.type	main, @function
main:
.LFB0:
	.cfi_startproc
	endbr64
	pushq	%rbp
	.cfi_def_cfa_offset 16
	.cfi_offset 6, -16
	movq	%rsp, %rbp
	.cfi_def_cfa_register 6
	subq	$16, %rsp
	movl	$0, -16(%rbp)
	movl	$1, -12(%rbp)
	movl	$10, %esi
	leaq	.LC0(%rip), %rax
	movq	%rax, %rdi
	movl	$0, %eax
	call	printf@PLT
\end{lstlisting}

\textbf{之后(fibonacci.o.asm的main函数部分):}
\begin{lstlisting}[language={[x86masm]Assembler}]
0000000000000000 <main>:
   0:	f3 0f 1e fa          	endbr64
   4:	55                   	push   %rbp
   5:	48 89 e5             	mov    %rsp,%rbp
   8:	48 83 ec 10          	sub    $0x10,%rsp
   c:	c7 45 f0 00 00 00 00 	movl   $0x0,-0x10(%rbp)
  13:	c7 45 f4 01 00 00 00 	movl   $0x1,-0xc(%rbp)
  1a:	be 0a 00 00 00       	mov    $0xa,%esi
  1f:	48 8d 05 00 00 00 00 	lea    0x0(%rip),%rax        # 26 <main+0x26>
  26:	48 89 c7             	mov    %rax,%rdi
  29:	b8 00 00 00 00       	mov    $0x0,%eax
  2e:	e8 00 00 00 00       	call   33 <main+0x33>
\end{lstlisting}

\subsection{链接器}

链接器将目标文件和库链接在一起,生成可执行文件。在GCC中,由ld工具执行。

链接器解析外部符号,将目标文件中的符号引用与库中的定义匹配。对于fibonacci.o,链接器链接了libc中的printf和putchar函数,生成了可执行文件fibonacci.exe。链接器还添加了启动代码(如\texttt{\_start}函数,调用main)、PLT(Procedure Linkage Table)用于动态链接。

fibonacci.exe.asm显示了完整的可执行文件反汇编,包括.init, .plt, .text, .fini段。PLT允许延迟绑定外部函数,提高加载效率。相比.o.asm,.exe.asm包含了所有依赖的库代码和重定位信息,使程序可执行。

对比fibonacci.o.asm和fibonacci.exe.asm:.o.asm只有main函数,而.exe.asm包含了整个程序的布局,包括库函数的入口点和动态链接机制。

\textbf{之前(fibonacci.o.asm的main函数部分):}
\begin{lstlisting}[language={[x86masm]Assembler}]
0000000000000000 <main>:
   0:	f3 0f 1e fa          	endbr64
   4:	55                   	push   %rbp
   5:	48 89 e5             	mov    %rsp,%rbp
   8:	48 83 ec 10          	sub    $0x10,%rsp
   c:	c7 45 f0 00 00 00 00 	movl   $0x0,-0x10(%rbp)
  13:	c7 45 f4 01 00 00 00 	movl   $0x1,-0xc(%rbp)
  1a:	be 0a 00 00 00       	mov    $0xa,%esi
  1f:	48 8d 05 00 00 00 00 	lea    0x0(%rip),%rax        # 26 <main+0x26>
  26:	48 89 c7             	mov    %rax,%rdi
  29:	b8 00 00 00 00       	mov    $0x0,%eax
  2e:	e8 00 00 00 00       	call   33 <main+0x33>
\end{lstlisting}

\textbf{之后(fibonacci.exe.asm的main函数部分):}
\begin{lstlisting}[language={[x86masm]Assembler}]
0000000000001169 <main>:
    1169:	f3 0f 1e fa          	endbr64
    116d:	55                   	push   %rbp
\end{lstlisting}

\subsection{编译器内部的过程}

预处理之后,编译过程进入核心阶段,代码将经历从高级抽象到低级机器指令的转变。GCC 使用了两种主要的中间表示(Intermediate Representation, IR)来完成这个过程:GIMPLE 和 RTL。

\subsubsection{词法分析:代码的“分词”}
编译器首先要像我们阅读文章一样,把一长串的源代码字符流“切分”成一个个有意义的基本单元,这些单元被称为**词法单元(Token)**。例如,`int a = 0;` 这行代码会被分解为:关键字 `int`、标识符 `a`、操作符 `=`、整型常量 `0` 和分隔符 `;`。

我们可以通过 Clang 的 `-dump-tokens` 命令得到 `tokens.txt`,其中记录了 `fibonacci.c` 被“分词”后的结果。下面是 `main` 函数开头部分的词法单元:
\begin{lstlisting}[language=text, caption={fibonacci.c 的部分词法单元 (tokens.txt)}]
int 'int'      [StartOfLine]  Loc=<fibonacci.c:5:1>
identifier 'main'      [LeadingSpace] Loc=<fibonacci.c:5:5>
l_paren '('             Loc=<fibonacci.c:5:9>
r_paren ')'             Loc=<fibonacci.c:5:10>
l_brace '{'      [LeadingSpace] Loc=<fibonacci.c:5:12>
int 'int'      [StartOfLine] [LeadingSpace] Loc=<fibonacci.c:6:5>
identifier 'a'       [LeadingSpace] Loc=<fibonacci.c:6:9>
equal '='      [LeadingSpace] Loc=<fibonacci.c:6:11>
numeric_constant '0'      [LeadingSpace] Loc=<fibonacci.c:6:13>
comma ','               Loc=<fibonacci.c:6:14>
\end{lstlisting}
每一行都清晰地标明了词法单元的类型(如 `keyword`、`identifier`)、内容(如 `'main'`)和它在源文件中的位置。

\subsubsection{语法分析:构建句法结构}
词法分析之后,编译器需要根据语言的语法规则,将离散的词法单元组织成一个具有层级结构的**抽象语法树(Abstract Syntax Tree, AST)**。这棵树精确地反映了代码的语法结构。如果代码存在语法错误(例如,`for` 循环缺少括号),这个阶段就会报错。

ast\_dump.txt 文件展示了 fibonacci.c 的 AST。树的根节点是TranslationUnitDecl(翻译单元声明),下面包含了我们定义的 main函数。
\begin{lstlisting}[language=text, caption={main 函数的抽象语法树片段 (ast\_dump.txt)}]
|-FunctionDecl 0x... <<fibonacci.c:5:1, line:22:1>> line:5:5 main 'int ()'
  `-CompoundStmt 0x... <line:5:12, line:22:1>
    |-DeclStmt 0x... <line:6:5, col:32>
    | `-VarDecl 0x... <col:5, col:9> col:9 used a 'int' cinit
    ...
    |-ForStmt 0x... <line:9:5, line:19:5>
    | |-DeclStmt 0x... <line:9:10, col:17>
    | | `-VarDecl 0x... <col:10, col:14> col:14 used i 'int' cinit
    | |-<<<NULL>>>
    | |-BinaryOperator 0x... <col:19, col:34> 'int' '<'
    ...
\end{lstlisting}
从中我们可以看到,`main` 函数 (`FunctionDecl`) 包含一个复合语句 (`CompoundStmt`),复合语句中又有一个 `ForStmt` 节点,清晰地还原了 `for` 循环的结构。

\subsubsection{语义分析与中间代码}
在 AST 的基础上,编译器进行**语义分析**,检查代码的逻辑是否自洽。这包括类型检查(例如,不能把一个字符串赋值给整型变量)、变量是否声明后才使用等。

通过所有检查后,编译器将 AST 转换为一种更接近机器指令的**中间表示(IR)**。GCC 使用 GIMPLE 和 RTL 作为其主要的 IR。这部分在之前的章节已有详细阐述,它作为优化的主要载体,对 GIMPLE 形式的代码进行常量传播、冗余消除等操作,然后转换为更低级的 RTL,为生成最终的汇编代码做准备。

\subsubsection{GIMPLE:接近源码的高级表示}

编译器首先将预处理后的代码转换成一种名为 **GIMPLE** 的高级中间表示。GIMPLE 的设计目标是既能保留部分源代码的结构(如循环和条件),又足够简单,以便进行各种与目标机器无关的代码优化。它采用的是一种“三地址码”的形式,即每个指令最多只涉及三个操作数。

在不进行任何优化(`-O0`)的情况下,`fibonacci.c` 的 `main` 函数转换的初始 GIMPLE 代码(位于 `fibonacci.c.004t.gimple`)如下所示。可以看到,代码结构与原始 C 代码非常相似,变量定义、循环结构和函数调用都清晰可见。

\begin{lstlisting}[language=C, caption={fibonacci.c.004t.gimple}]
;; Function main (main, funcdef_no=0, decl_uid=2095, cgraph_uid=1, symbol_order=0)

main ()
{
  int next;
  int b;
  int a;
  int i;
  int D.2104;

  <bb 2> [local count: 1073741824]:
  a = 0;
  b = 1;
  printf ("Fibonacci Series up to %d terms:\n", 10);
  i = 0;
  goto <bb 4>; [100.00%]

  <bb 3> [local count: 976367641]:
  if (i <= 1)
    {
      next = i;
    }
  else
    {
      next = a + b;
      a = b;
      b = next;
    }
  printf ("%d ", next);
  i = i + 1;

  <bb 4> [local count: 1073741824]:
  if (i <= 9)
    {
      goto <bb 3>; [91.00%]
    }
  else
    {
      goto <bb 5>; [9.00%]
    }

  <bb 5> [local count: 97374183]:
  printf ("\n");
  D.2104 = 0;
  return D.2104;

}
\end{lstlisting}

\subsubsection{从接近高级语言的GIMPLE到接近机器语言的RTL}

GIMPLE 的真正威力在于它是绝大多数优化的载体。当开启优化选项(如 `-O2`)时,编译器会执行数百个优化过程(Pass),对 GIMPLE 代码进行分析和转换,以期生成更高效的代码。常见的优化手段包括:
\begin{itemize}
    \item \textbf{常量传播:} 将常量值直接替换到使用它的地方。
    \item \textbf{冗余消除:} 删除重复的计算。
    \item \textbf{循环优化:} 如循环展开、循环不变代码外提等。
    \item \textbf{函数内联:} 将小函数的调用直接替换为函数体本身。
\end{itemize}
经过 `-O2` 优化后,虽然初始 GIMPLE 变化不大,但在后续的优化传递中,代码结构会被极大改变,为生成更高效的底层代码铺平道路。

\subsubsection{RTL:面向机器的低级表示}

当 GIMPLE 阶段的优化完成后,编译器会将其转换为一种更低级的、更接近机器指令的中间表示——**RTL (Register Transfer Language)**。RTL 描述了数据如何在寄存器之间传送和计算,它为目标机器相关的优化(如指令选择和寄存器分配)提供了基础。

RTL 代码看起来更像汇编语言的抽象描述。以下是 `-O0`(未优化)和 `-O2`(优化)下 `main` 函数最终生成的 RTL 代码(位于 `*.final` 文件)的片段对比。

\textbf{未优化的 Final RTL (`-O0`)}:
代码显得冗长,频繁地在栈(`[rbp-...]`)和寄存器之间移动数据。这是因为 `-O0` 旨在直接翻译代码,而不关心性能。
\begin{lstlisting}[language=text, caption={fibonacci.c.273r.final (未优化片段)}]
(insn 15 14 16 2 (set (reg:SI 91 [ i ])
        (const_int 0 [0])) "fibonacci.c":9:5 -1
     (nil))
(jump_insn 16 15 17 2 (set (pc)
        (label_ref 29)) "fibonacci.c":9:5 -1
     (nil)
 -> 29)
(insn 22 21 23 2 (set (reg:SI 88 [ next ])
        (reg:SI 91 [ i ])) "fibonacci.c":11:13 -1
     (nil))
(jump_insn 23 22 24 2 (set (pc)
        (label_ref 27)) "fibonacci.c":11:13 -1
     (nil)
 -> 27)
\end{lstlisting}

\textbf{优化后的 Final RTL (`-O2`)}:
代码精简了许多。编译器通过优化,将许多变量直接保存在寄存器中,减少了内存访问。例如,循环计数器和斐波那契数列的值可能长时间驻留在寄存器中,大大提高了执行效率。
\begin{lstlisting}[language=text, caption={fibonacci.c.340r.final (优化片段)}]
(insn 7 6 8 2 (set (reg:SI 91)
        (const_int 1 [0x1])) "fibonacci.c":6:20 -1
     (nil))
(insn 8 7 9 2 (set (reg:SI 92)
        (const_int 0 [0])) "fibonacci.c":6:20 -1
     (nil))
(insn 9 8 10 2 (set (reg:SI 93)
        (const_int 0 [0])) "fibonacci.c":9:5 -1
     (nil))
(jump_insn 10 9 11 2 (set (pc)
        (label_ref 21)) "fibonacci.c":9:5 -1
     (nil)
 -> 21)
\end{lstlisting}

最终,这些 RTL 指令会被转换成目标平台的汇编代码,完成整个编译过程。通过观察从 GIMPLE 到 RTL 的演变,以及不同优化级别下的差异,我们可以深刻体会到编译器是如何在忠实于原始逻辑的基础上,对代码进行精雕细琢,以追求极致性能的。
\section{第二题:汇编代码撰写}

第二题中,王泽舜负责编写与源程序等价的llvm IR 代码,杨乔钦负责编写与源程序等价的riscv汇编代码。
\begin{lstlisting}[language=text,caption={test\_new.ll}]
; 目标平台信息,消除编译警告
target triple = "x86_64-pc-linux-gnu"
; 全局常量和数组
@N = constant i32 5
@global_array = global [5 x i32] [i32 2, i32 2, i32 3, i32 4, i32 5]
; 外部函数声明
declare void @putint(i32)
declare void @putch(i32)
; main 函数
define i32 @main() {
entry:
    ; 局部变量初始化
    %sum = alloca i32
    %i = alloca i32
    %count = alloca i32
    
    store i32 0, ptr %sum        ; sum = 0
    store i32 0, ptr %i          ; i = 0
    store i32 5, ptr %count      ; count = 5
    
    br label %while_check

while_check:
    ; 检查循环条件: i < count
    %i_val = load i32, ptr %i
    %count_val = load i32, ptr %count
    %cmp = icmp slt i32 %i_val, %count_val
    br i1 %cmp, label %while_body, label %while_end

while_body:
    ; 循环体: sum = sum + global_array[i]
    %sum_val = load i32, ptr %sum
    %i_curr = load i32, ptr %i
    
    ; 计算 global_array[i] 的地址
    %array_ptr = getelementptr [5 x i32], ptr @global_array, i32 0, i32 %i_curr
    %array_val = load i32, ptr %array_ptr
    
    ; sum = sum + global_array[i]
    %new_sum = add i32 %sum_val, %array_val
    store i32 %new_sum, ptr %sum
    
    ; i = i + 1
    %new_i = add i32 %i_curr, 1
    store i32 %new_i, ptr %i
    
    br label %while_check

while_end:
    ; if (sum > 20)
    %final_sum = load i32, ptr %sum
    %if_cmp = icmp sgt i32 %final_sum, 20
    br i1 %if_cmp, label %then_branch, label %else_branch

then_branch:
    ; int result = calculate(sum, 2);
    %result = call i32 @calculate(i32 %final_sum, i32 2)
    
    ; putint(result); putch(10);
    call void @putint(i32 %result)
    call void @putch(i32 10)
    br label %exit

else_branch:
    ; int remainder = sum % 3;
    %remainder = srem i32 %final_sum, 3
    
    ; putint(remainder); putch(10);
    call void @putint(i32 %remainder)
    call void @putch(i32 10)
    br label %exit

exit:
    ret i32 0
}

; calculate 函数
define i32 @calculate(i32 %a, i32 %b) {
entry:
    ; 局部变量
    %result = alloca i32
    %b_local = alloca i32
    
    store i32 0, ptr %result     ; result = 0
    store i32 %b, ptr %b_local   ; 保存参数 b
    
    br label %loop_check

loop_check:
    ; 检查循环条件: b > 0
    %b_val = load i32, ptr %b_local
    %loop_cmp = icmp sgt i32 %b_val, 0
    br i1 %loop_cmp, label %loop_body, label %loop_end

loop_body:
    ; result = result + a
    %result_val = load i32, ptr %result
    %new_result = add i32 %result_val, %a
    store i32 %new_result, ptr %result
    
    ; b = b - 1
    %b_curr = load i32, ptr %b_local
    %new_b = sub i32 %b_curr, 1
    store i32 %new_b, ptr %b_local
    
    br label %loop_check

loop_end:
    %final_result = load i32, ptr %result
    ret i32 %final_result
}

\end{lstlisting}

% \newpage
% \bibliographystyle{plain}
% \bibliography{reference.bib} 

\end{document}